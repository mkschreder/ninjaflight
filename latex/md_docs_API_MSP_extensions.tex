Cleanflight includes a number of extensions to the Multi\+Wii Serial Protocol (M\+S\+P). This document describes those extensions in order that 3rd party tools may identify cleanflight firmware and react appropriately.

Issue the M\+S\+P\+\_\+\+A\+P\+I\+\_\+\+V\+E\+R\+S\+I\+O\+N command to find out if the firmware supports them.

\subsection*{Mode Ranges}

\subsubsection*{M\+S\+P\+\_\+\+M\+O\+D\+E\+\_\+\+R\+A\+N\+G\+E\+S}

The M\+S\+P\+\_\+\+M\+O\+D\+E\+\_\+\+R\+A\+N\+G\+E\+S returns the current auxiliary mode settings from the flight controller. It should be invoked before any modification is made to the configuration.

The message returns a group of 4 unsigned bytes for each 'slot' available in the flight controller. The number of slots should be calculated from the size of the returned message.

\begin{TabularC}{4}
\hline
\rowcolor{lightgray}{\bf Command }&{\bf Msg Id }&{\bf Direction }&{\bf Notes  }\\\cline{1-4}
M\+S\+P\+\_\+\+M\+O\+D\+E\+\_\+\+R\+A\+N\+G\+E\+S &34 &to F\+C &Following this command, the F\+C returns a block of 4 bytes for each auxiliary mode 'slot' \\\cline{1-4}
\end{TabularC}
Unassigned slots have range\+Start\+Step == range\+End\+Step. Each element contains the following fields.

\begin{TabularC}{3}
\hline
\rowcolor{lightgray}{\bf Data }&{\bf Type }&{\bf Notes  }\\\cline{1-3}
permanent\+Id &uint8 &See \hyperlink{Modes_8md}{Modes.\+md} for a definition of the permanent ids \\\cline{1-3}
aux\+Channel\+Index &uint8 &The Aux switch number (indexed from 0) \\\cline{1-3}
range\+Start\+Step &uint8 &The start value for this element in 'blocks' of 25 where 0 == 900 and 48 == 2100 \\\cline{1-3}
range\+End\+Step &uint8 &The end value for this element in 'blocks' of 25 where 0 == 900 and 48 == 2100 \\\cline{1-3}
\end{TabularC}
Thus, for a cleanflight firmware with 40 slots 160 bytes would be returned in response to M\+S\+P\+\_\+\+M\+O\+D\+E\+\_\+\+R\+A\+N\+G\+E\+S,

\subsubsection*{M\+S\+P\+\_\+\+S\+E\+T\+\_\+\+M\+O\+D\+E\+\_\+\+R\+A\+N\+G\+E}

The M\+S\+P\+\_\+\+S\+E\+T\+\_\+\+M\+O\+D\+E\+\_\+\+R\+A\+N\+G\+E is used to inform the flight controller of auxiliary mode settings. The client {\itshape must} return all auxiliary elements, including those that have been disabled or are undefined, by sending this message for all auxiliary slots.

\begin{TabularC}{3}
\hline
\rowcolor{lightgray}{\bf Command }&{\bf Msg Id }&{\bf Direction  }\\\cline{1-3}
M\+S\+P\+\_\+\+S\+E\+T\+\_\+\+M\+O\+D\+E\+\_\+\+R\+A\+N\+G\+E &35 &to F\+C \\\cline{1-3}
\end{TabularC}


\begin{TabularC}{3}
\hline
\rowcolor{lightgray}{\bf Data }&{\bf Type }&{\bf Notes  }\\\cline{1-3}
sequence id &uint8 &A monotonically increasing I\+D, from 0 to the number of slots -\/1 \\\cline{1-3}
permanent\+Id &uint8 &See \hyperlink{Modes_8md}{Modes.\+md} for a definition of the permanent ids \\\cline{1-3}
aux\+Channel\+Index &uint8 &The Aux channel number (indexed from 0) \\\cline{1-3}
range\+Start\+Step &uint8 &The start value for this element in 'blocks' of 25 where 0 == 900 and 48 == 2100 \\\cline{1-3}
range\+End\+Step &uint8 &The end value for this element in 'blocks' of 25 where 0 == 900 and 48 == 2100 \\\cline{1-3}
\end{TabularC}
\subsubsection*{Implementation Notes}


\begin{DoxyItemize}
\item The client should make no assumptions about the number of slots available. Rather, the number should be computed from the size of the M\+S\+P\+\_\+\+M\+O\+D\+E\+\_\+\+R\+A\+N\+G\+E\+S message divided by the size of the returned data element (4 bytes);
\item The client should ensure that all changed items are returned to the flight controller, including those where a switch or range has been disabled;
\item A 'null' return, with all values other than the sequence id set to 0, must be made for all unused slots, up to the maximum number of slots calculated from the initial message.
\end{DoxyItemize}

\subsection*{Adjustment Ranges}

\subsubsection*{M\+S\+P\+\_\+\+A\+D\+J\+U\+S\+T\+M\+E\+N\+T\+\_\+\+R\+A\+N\+G\+E\+S}

The M\+S\+P\+\_\+\+A\+D\+J\+U\+S\+T\+M\+E\+N\+T\+\_\+\+R\+A\+N\+G\+E\+S returns the current adjustment range settings from the flight controller. It should be invoked before any modification is made to the configuration.

The message returns a group of 6 unsigned bytes for each 'slot' available in the flight controller. The number of slots should be calculated from the size of the returned message.

\begin{TabularC}{4}
\hline
\rowcolor{lightgray}{\bf Command }&{\bf Msg Id }&{\bf Direction }&{\bf Notes  }\\\cline{1-4}
M\+S\+P\+\_\+\+A\+D\+J\+U\+S\+T\+M\+E\+N\+T\+\_\+\+R\+A\+N\+G\+E\+S &52 &to F\+C &Following this command, the F\+C returns a block of 6 bytes for each adjustment range 'slot' \\\cline{1-4}
\end{TabularC}
Unassigned slots have range\+Start\+Step == range\+End\+Step. Each element contains the following fields.

\begin{TabularC}{3}
\hline
\rowcolor{lightgray}{\bf Data }&{\bf Type }&{\bf Notes  }\\\cline{1-3}
adjustment\+State\+Index &uint8 &See below \\\cline{1-3}
aux\+Channel\+Index &uint8 &The Aux channel number (indexed from 0) used to activate the adjustment \\\cline{1-3}
range\+Start\+Step &uint8 &The start value for this element in 'blocks' of 25 where 0 == 900 and 48 == 2100 \\\cline{1-3}
range\+End\+Step &uint8 &The end value for this element in 'blocks' of 25 where 0 == 900 and 48 == 2100 \\\cline{1-3}
adjustment\+Function &uint8 &See below \\\cline{1-3}
aux\+Switch\+Channel\+Index &uint8 &The Aux channel number used to perform the function (indexed from 0) \\\cline{1-3}
\end{TabularC}
Thus, for a cleanflight firmware with 12 slots 72 bytes would be returned in response to M\+S\+P\+\_\+\+A\+D\+J\+U\+S\+T\+M\+E\+N\+T\+\_\+\+R\+A\+N\+G\+E\+S,

\subsubsection*{M\+S\+P\+\_\+\+S\+E\+T\+\_\+\+A\+D\+J\+U\+S\+T\+M\+E\+N\+T\+\_\+\+R\+A\+N\+G\+E}

The M\+S\+P\+\_\+\+S\+E\+T\+\_\+\+A\+D\+J\+U\+S\+T\+M\+E\+N\+T\+\_\+\+R\+A\+N\+G\+E is used to inform the flight controller of adjustment range settings. The client {\itshape must} return all adjustment range elements, including those that have been disabled or are undefined, by sending this message for all adjustment range slots.

\begin{TabularC}{3}
\hline
\rowcolor{lightgray}{\bf Command }&{\bf Msg Id }&{\bf Direction  }\\\cline{1-3}
M\+S\+P\+\_\+\+S\+E\+T\+\_\+\+A\+D\+J\+U\+S\+T\+M\+E\+N\+T\+\_\+\+R\+A\+N\+G\+E &53 &to F\+C \\\cline{1-3}
\end{TabularC}


\begin{TabularC}{3}
\hline
\rowcolor{lightgray}{\bf Data }&{\bf Type }&{\bf Notes  }\\\cline{1-3}
sequence id &uint8 &A monotonically increasing I\+D, from 0 to the number of slots -\/1 \\\cline{1-3}
adjustment\+State\+Index &uint8 &See below \\\cline{1-3}
aux\+Channel\+Index &uint8 &The Aux channel number (indexed from 0) \\\cline{1-3}
range\+Start\+Step &uint8 &The start value for this element in 'blocks' of 25 where 0 == 900 and 48 == 2100 \\\cline{1-3}
range\+End\+Step &uint8 &The end value for this element in 'blocks' of 25 where 0 == 900 and 48 == 2100 \\\cline{1-3}
adjustment\+Function &uint8 &See below \\\cline{1-3}
aux\+Switch\+Channel\+Index &uint8 &The Aux channel number used to perform the function (indexed from 0) \\\cline{1-3}
\end{TabularC}
\subsubsection*{M\+S\+P\+\_\+\+S\+E\+T\+\_\+1\+W\+I\+R\+E}

The M\+S\+P\+\_\+\+S\+E\+T\+\_\+1\+W\+I\+R\+E is used to enable serial1wire passthrough note\+: it would be ideal to disable this when armed

\begin{TabularC}{3}
\hline
\rowcolor{lightgray}{\bf Command }&{\bf Msg Id }&{\bf Direction  }\\\cline{1-3}
M\+S\+P\+\_\+\+S\+E\+T\+\_\+1\+W\+I\+R\+E &243 &to F\+C \\\cline{1-3}
\end{TabularC}
\begin{TabularC}{3}
\hline
\rowcolor{lightgray}{\bf Data }&{\bf Type }&{\bf Notes  }\\\cline{1-3}
esc id &uint8 &A monotonically increasing I\+D, from 0 to the number of escs -\/1 \\\cline{1-3}
\end{TabularC}
\paragraph*{Adjustment\+Index}

The F\+C maintains internal state for each adjustment\+State\+Index, currently 4 simultaneous adjustment states are maintained. Multiple adjustment ranges can be configured to use the same state but care should be taken not to send multiple adjustment ranges that when active would confict.

e.\+g. Configuring two identical adjustment ranges using the same slot would conflict, but configuring two adjustment ranges that used only one half of the possible channel range each but used the same adjustment\+State\+Index would not conflict.

The F\+C does N\+O\+T check for conflicts.

\paragraph*{Adjustment\+Function}

There are many adjustments that can be made, the numbers of them and their use is found in the documentation of the cli {\ttfamily adjrange} command in the 'Inflight Adjustents' section.

\subsubsection*{Implementation Notes}


\begin{DoxyItemize}
\item The client should make no assumptions about the number of slots available. Rather, the number should be computed from the size of the M\+S\+P\+\_\+\+A\+D\+J\+U\+S\+T\+M\+E\+N\+T\+\_\+\+R\+A\+N\+G\+E\+S message divided by the size of the returned data element (6 bytes);
\item The client should ensure that all changed items are returned to the flight controller, including those where a switch or range has been disabled;
\item A 'null' return, with all values except for the sequence id set to 0, must be made for all unused slots, up to the maximum number of slots calculated from the initial message.
\end{DoxyItemize}

\subsection*{Deprecated M\+S\+P}

The following M\+S\+P commands are replaced by the M\+S\+P\+\_\+\+M\+O\+D\+E\+\_\+\+R\+A\+N\+G\+E\+S and M\+S\+P\+\_\+\+S\+E\+T\+\_\+\+M\+O\+D\+E\+\_\+\+R\+A\+N\+G\+E extensions, and are not recognised by cleanflight.


\begin{DoxyItemize}
\item M\+S\+P\+\_\+\+B\+O\+X
\item M\+S\+P\+\_\+\+S\+E\+T\+\_\+\+B\+O\+X
\end{DoxyItemize}

\subsection*{See also }

\hyperlink{Modes_8md}{Modes.\+md} describes the user visible implementation for the cleanflight modes extension. 