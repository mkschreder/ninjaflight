Cleanflight has a battery monitoring feature. The voltage of the main battery can be measured by the system and used to trigger a low-\/battery warning \hyperlink{Buzzer_8md}{buzzer}, on-\/board status L\+E\+D flashing and L\+E\+D strip patterns.

Low battery warnings can\+:


\begin{DoxyItemize}
\item Help ensure you have time to safely land the aircraft
\item Help maintain the life and safety of your Li\+Po/\+Li\+Fe batteries, which should not be discharged below manufacturer recommendations
\end{DoxyItemize}

Minimum and maximum cell voltages can be set, and these voltages are used to auto-\/detect the number of cells in the battery when it is first connected.

Per-\/cell monitoring is not supported, as we only use one A\+D\+C to read the battery voltage.

\subsection*{Supported targets}

All targets support battery voltage monitoring unless status.

\subsection*{Connections}

When dealing with batteries {\bfseries A\+L\+W\+A\+Y\+S C\+H\+E\+C\+K P\+O\+L\+A\+R\+I\+T\+Y!}

Measure expected voltages {\bfseries first} and then connect to the flight controller. Powering the flight controller with incorrect voltage or reversed polarity will likely fry your flight controller. Ensure your flight controller has a voltage divider capable of measuring your particular battery voltage.

\subsubsection*{Naze32}

The Naze32 has an on-\/board battery divider circuit; just connect your main battery to the V\+B\+A\+T connector.

{\bfseries C\+A\+U\+T\+I\+O\+N\+:} When installing the connection from main battery to the V\+B\+A\+T connector, be sure to first disconnect the main battery from the frame/power distribution board. Check the wiring very carefully before connecting battery again. Incorrect connections can immediately and completely destroy the flight controller and connected peripherals (E\+S\+C, G\+P\+S, Receiver etc.).

\subsubsection*{C\+C3\+D}

The C\+C3\+D has no battery divider. To use voltage monitoring, you must create a divider that gives a 3.\+3v M\+A\+X\+I\+M\+U\+M output when the main battery is fully charged. Connect the divider output to S5\+\_\+\+I\+N/\+P\+A0/\+R\+C5.

Notes\+:


\begin{DoxyItemize}
\item S5\+\_\+\+I\+N/\+P\+A0/\+R\+C5 is Pin 7 on the 8 pin connector, second to last pin, on the opposite end from the G\+N\+D/+5/\+P\+P\+M signal input.
\item When battery monitoring is enabled on the C\+C3\+D, R\+C5 can no-\/longer be used for P\+W\+M input.
\end{DoxyItemize}

\subsubsection*{Sparky}

See the Board -\/ Sparky.\+md \char`\"{}\+Sparky board chapter\char`\"{}.

\subsection*{Configuration}

Enable the {\ttfamily V\+B\+A\+T} feature.

Configure min/max cell voltages using the following C\+L\+I setting\+:

{\ttfamily vbat\+\_\+scale} -\/ Adjust this to match actual measured battery voltage to reported value.

{\ttfamily vbat\+\_\+max\+\_\+cell\+\_\+voltage} -\/ Maximum voltage per cell, used for auto-\/detecting battery voltage in 0.\+1\+V units, i.\+e. 43 = 4.\+3\+V

{\ttfamily set vbat\+\_\+warning\+\_\+cell\+\_\+voltage} -\/ Warning voltage per cell; this triggers battery-\/out alarms, in 0.\+1\+V units, i.\+e. 34 = 3.\+4\+V

{\ttfamily vbat\+\_\+min\+\_\+cell\+\_\+voltage} -\/ Minimum voltage per cell; this triggers battery-\/out alarms, in 0.\+1\+V units, i.\+e. 33 = 3.\+3\+V

e.\+g.

``` set vbat\+\_\+scale = 110 set vbat\+\_\+max\+\_\+cell\+\_\+voltage = 43 set vbat\+\_\+warning\+\_\+cell\+\_\+voltage = 34 set vbat\+\_\+min\+\_\+cell\+\_\+voltage = 33 ```

\section*{Current Monitoring}

Current monitoring (amperage) is supported by connecting a current meter to the appropriate current meter A\+D\+C input (see the documentation for your particular board).

When enabled, the following values calculated and used by the telemetry and O\+L\+E\+D display subsystems\+:
\begin{DoxyItemize}
\item Amps
\item m\+Ah used
\item Capacity remaining
\end{DoxyItemize}

\subsection*{Configuration}

Enable current monitoring using the C\+L\+I command\+:

``` feature C\+U\+R\+R\+E\+N\+T\+\_\+\+M\+E\+T\+E\+R ```

Configure the current meter type using the {\ttfamily current\+\_\+meter\+\_\+type} settings here\+:

\begin{TabularC}{2}
\hline
\rowcolor{lightgray}{\bf Value }&{\bf Sensor Type  }\\\cline{1-2}
N\+O\+N\+E &None \\\cline{1-2}
A\+D\+C &A\+D\+C/hardware sensor \\\cline{1-2}
V\+I\+R\+T\+U\+A\+L &Virtual sensor \\\cline{1-2}
\end{TabularC}
Configure capacity using the {\ttfamily battery\+\_\+capacity} setting, in m\+Ah units.

If you're using an O\+S\+D that expects the multiwii current meter output value, then set {\ttfamily multiwii\+\_\+current\+\_\+meter\+\_\+output} to {\ttfamily O\+N} (this multiplies amperage sent to M\+S\+P by 10 and truncates negative values)).

\subsubsection*{A\+D\+C Sensor}

The current meter may need to be configured so the value read at the A\+D\+C input matches actual current draw. Just like you need a voltmeter to correctly calibrate your voltage reading you also need an ammeter to calibrate the current sensor.

Use the following settings to adjust calibration\+:

{\ttfamily current\+\_\+meter\+\_\+scale} {\ttfamily current\+\_\+meter\+\_\+offset}

It is recommended to set {\ttfamily multiwii\+\_\+current\+\_\+meter\+\_\+output} to {\ttfamily O\+F\+F} when calibrating A\+D\+C current sensor.

\subsubsection*{Virtual Sensor}

The virtual sensor uses the throttle position to calculate an estimated current value. This is useful when a real sensor is not available. The following settings adjust the virtual sensor calibration\+:

\begin{TabularC}{2}
\hline
\rowcolor{lightgray}{\bf Setting }&{\bf Description  }\\\cline{1-2}
{\ttfamily current\+\_\+meter\+\_\+scale} &The throttle scaling factor \mbox{[}centiamps, i.\+e. 1/100th A\mbox{]} \\\cline{1-2}
{\ttfamily current\+\_\+meter\+\_\+offset} &The current at zero throttle (while disarmed) \mbox{[}centiamps, i.\+e. 1/100th A\mbox{]} \\\cline{1-2}
\end{TabularC}
There are two simple methods to tune these parameters\+: one uses a battery charger and another depends on actual current measurements.

\paragraph*{Tuning Using Actual Current Measurements}

If you know your craft's current draw (in Amperes) while disarmed (Imin) and at maximum throttle while armed (Imax), calculate the scaling factors as follows\+: ``` current\+\_\+meter\+\_\+scale = (Imax -\/ Imin) $\ast$ 100000 / (Tmax + (Tmax $\ast$ Tmax / 50)) current\+\_\+meter\+\_\+offset = Imin $\ast$ 100 ``{\ttfamily  Note\+: Tmax is maximum throttle offset (i.\+e. for}max\+\_\+throttle` = 1850, Tmax = 1850 -\/ 1000 = 850)

For example, assuming a maximum current of 34.\+2\+A, a minimum current of 2.\+8\+A, and a Tmax {\ttfamily max\+\_\+throttle} = 1850\+: ``` current\+\_\+meter\+\_\+scale = (Imax -\/ Imin) $\ast$ 100000 / (Tmax + (Tmax $\ast$ Tmax / 50)) = (34.\+2 -\/ 2.\+8) $\ast$ 100000 / (850 + (850 $\ast$ 850 / 50)) = 205 current\+\_\+meter\+\_\+offset = Imin $\ast$ 100 = 280 ``` \paragraph*{Tuning Using Battery Charger Measurement}

If you cannot measure current draw directly, you can approximate it indirectly using your battery charger. However, note it may be difficult to adjust {\ttfamily current\+\_\+meter\+\_\+offset} using this method unless you can measure the actual current draw with the craft disarmed.

Note\+:
\begin{DoxyItemize}
\item This method depends on the accuracy of your battery charger; results may vary.
\item If you add or replace equipment that changes the in-\/flight current draw (e.\+g. video transmitter, camera, gimbal, motors, prop pitch/sizes, E\+S\+Cs, etc.), you should recalibrate.
\end{DoxyItemize}

The general method is\+:


\begin{DoxyEnumerate}
\item Fully charge your flight battery
\item Fly your craft, using $>$50\% of your battery pack capacity (estimated)
\item Note Cleanflight's reported m\+Ah draw
\item Re-\/charge your flight battery, noting the m\+Ah charging data needed to restore the pack to fully charged
\item Adjust {\ttfamily current\+\_\+meter\+\_\+scale} to according to the formula given below
\item Repeat and test
\end{DoxyEnumerate}

Given (a) the reported m\+Ah draw and the (b) m\+Ah charging data, calculate a new {\ttfamily current\+\_\+meter\+\_\+scale} value as follows\+: ``` current\+\_\+meter\+\_\+scale = (charging\+\_\+data\+\_\+m\+Ah / reported\+\_\+draw\+\_\+m\+Ah) $\ast$ old\+\_\+current\+\_\+meter\+\_\+scale ``` For example, assuming\+:
\begin{DoxyItemize}
\item A Cleanflight reported current draw of 1260 m\+Ah
\item Charging data to restore full charge of 1158 m\+Ah
\item A existing {\ttfamily current\+\_\+meter\+\_\+scale} value of 400 (the default)
\end{DoxyItemize}

Then the updated {\ttfamily current\+\_\+meter\+\_\+scale} is\+: ``` current\+\_\+meter\+\_\+scale = (charging\+\_\+data\+\_\+m\+Ah / reported\+\_\+draw\+\_\+m\+Ah) $\ast$ old\+\_\+current\+\_\+meter\+\_\+scale = (1158 / 1260) $\ast$ 400 = 368 ``` 