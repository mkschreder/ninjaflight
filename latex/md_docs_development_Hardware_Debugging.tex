The code can be compiled with debugging information, you can then upload a debug version to a board via a J\+Link/\+St-\/\+Link debug adapter and step through the code in your I\+D\+E.

More information about the necessary hardware and setting up the eclipse I\+D\+E can be found Hardware Debugging in Eclipse.\+md \char`\"{}here\char`\"{}

A guide for visual studio can be found here\+: \href{http://visualgdb.com/tutorials/arm/st-link/}{\tt http\+://visualgdb.\+com/tutorials/arm/st-\/link/}

This video is also helpful in understanding the proces\+: \href{https://www.youtube.com/watch?v=kjvqySyNw20}{\tt https\+://www.\+youtube.\+com/watch?v=kjvqy\+Sy\+Nw20}

\subsection*{Hardware}

Various debugging hardware solutions exist, the Segger J-\/\+Link clones are cheap and are known to work on Windows with both the Naze and Olimexino platforms.

\subsubsection*{J-\/\+Link devices}

Segger make excellent debuggers and debug software.

The Segger J-\/\+Link G\+D\+B server can be obtained from here.

\href{http://www.segger.com/jlink-software.html}{\tt http\+://www.\+segger.\+com/jlink-\/software.\+html}

\paragraph*{Segger J-\/\+Link E\+D\+U E\+D\+U version, for hobbyists and educational use.}



\href{https://www.segger.com/j-link-edu.html}{\tt https\+://www.\+segger.\+com/j-\/link-\/edu.\+html}

\paragraph*{U\+S\+B-\/\+Mini\+J\+T\+A\+G J-\/\+Link J\+T\+A\+G/\+S\+W\+D Debugger/\+Emulator}

\href{http://www.hotmcu.com/usbminijtag-jlink-jtagswd-debuggeremula%E2%80%8Btor-p-29.html?cPath=3_25&zenid=fdefvpnod186umrhsek225dc10}{\tt http\+://www.\+hotmcu.\+com/usbminijtag-\/jlink-\/jtagswd-\/debuggeremula\%\+E2\%80\%8\+Btor-\/p-\/29.\+html?c\+Path=3\+\_\+25\&zenid=fdefvpnod186umrhsek225dc10}



\subparagraph*{A\+R\+M-\/\+J\+T\+A\+G-\/20-\/10 adapter}

\href{https://www.olimex.com/Products/ARM/JTAG/ARM-JTAG-20-10/}{\tt https\+://www.\+olimex.\+com/\+Products/\+A\+R\+M/\+J\+T\+A\+G/\+A\+R\+M-\/\+J\+T\+A\+G-\/20-\/10/} \href{http://uk.farnell.com/jsp/search/productdetail.jsp?sku=2144328}{\tt http\+://uk.\+farnell.\+com/jsp/search/productdetail.\+jsp?sku=2144328}



\paragraph*{C\+J\+M\+C\+U-\/\+S\+T\+M32 Singlechip Development Board Jlink Downloader Jlink A\+R\+M Programmer}





\href{http://www.goodluckbuy.com/cjmcu-stm32-singlechip-development-board-jlink-downloader-jlink-arm-programmer.html}{\tt http\+://www.\+goodluckbuy.\+com/cjmcu-\/stm32-\/singlechip-\/development-\/board-\/jlink-\/downloader-\/jlink-\/arm-\/programmer.\+html}

\subsubsection*{S\+T\+Link V2 devices}

S\+T\+Link V2 devices can be used too, via Open\+O\+C\+D.

\paragraph*{C\+E\+Park S\+T\+Link V2}



\href{http://www.goodluckbuy.com/cepark-stlink-st-link-v2-emulator-programmer-stm8-stm32-downloader.html}{\tt http\+://www.\+goodluckbuy.\+com/cepark-\/stlink-\/st-\/link-\/v2-\/emulator-\/programmer-\/stm8-\/stm32-\/downloader.\+html}

\subsection*{Compilation options}

use {\ttfamily D\+E\+B\+U\+G=G\+D\+B} make argument.

You may find that if you compile all the files with debug information on that the program is too big to fit on the target device. If this happens you have some options\+:


\begin{DoxyItemize}
\item Compile all files without debug information ({\ttfamily make clean}, {\ttfamily make ...}), then re-\/save or {\ttfamily touch} the files you want to be able to step though and then run {\ttfamily make D\+E\+B\+U\+G=G\+D\+B}. This will then re-\/compile the files you're interested in debugging with debugging symbols and you will get a smaller binary file which should then fit on the device.
\item You could use a development board such as an Olimexino or an E\+U\+S\+T\+M32\+F103\+R\+B, development boards often have more flash rom.
\end{DoxyItemize}

\subsection*{O\+S\+X}

\subsubsection*{Install Open\+O\+C\+D via Brew}

ruby -\/e \char`\"{}\$(curl -\/fs\+S\+L https\+://raw.\+githubusercontent.\+com/\+Homebrew/install/master/install)\char`\"{}

brew install openocd

\subsubsection*{G\+D\+B debug server}

\paragraph*{J-\/\+Link}

\subparagraph*{Windows}

Run the Launch the J-\/\+Link G\+D\+B Server program and configure using U\+I.

\paragraph*{Open\+O\+C\+D}

\subparagraph*{Windows}

S\+T\+M32\+F103 targets \begin{DoxyVerb}"C:\Program Files (x86)\UTILS\openocd-0.8.0\bin-x64\openocd-x64-0.8.0.exe" -f interface/stlink-v2.cfg -f target/stm32f1x_stlink.cfg
\end{DoxyVerb}


S\+T\+M32\+F30x targets \begin{DoxyVerb}"C:\Program Files (x86)\UTILS\openocd-0.8.0\bin-x64\openocd-x64-0.8.0.exe" -f scripts\board\stm32f3discovery.cfg
\end{DoxyVerb}


\subparagraph*{O\+S\+X/\+Linux}

S\+T\+M32\+F30x targets \begin{DoxyVerb}  openocd -f /usr/share/openocd/scripts/board/stm32vldiscovery.cfg\end{DoxyVerb}
 