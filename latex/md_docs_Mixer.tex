Cleanflight supports a number of mixing configurations as well as custom mixing. Mixer configurations determine how the servos and motors work together to control the aircraft.

\subsection*{Configuration}

To use a built-\/in mixing configuration, you can use the Chrome configuration G\+U\+I. It includes images of the various mixer types to assist in making the proper connections. See the Configuration section of the documentation for more information on the G\+U\+I.

You can also use the Command Line Interface (C\+L\+I) to set the mixer type\+:


\begin{DoxyEnumerate}
\item Use {\ttfamily mixer list} to see a list of supported mixes
\item Select a mixer. For example, to select T\+R\+I, use {\ttfamily mixer T\+R\+I}
\item You must use {\ttfamily save} to preserve your changes
\end{DoxyEnumerate}

\subsection*{Supported Mixer Types}

\begin{TabularC}{4}
\hline
\rowcolor{lightgray}{\bf Name }&{\bf Description }&{\bf Motors }&{\bf Servos  }\\\cline{1-4}
T\+R\+I &Tricopter &M1-\/\+M3 &S1 \\\cline{1-4}
Q\+U\+A\+D\+P &Quadcopter-\/\+Plus &M1-\/\+M4 &None \\\cline{1-4}
Q\+U\+A\+D\+X &Quadcopter-\/\+X &M1-\/\+M4 &None \\\cline{1-4}
B\+I &Bicopter (left/right) &M1-\/\+M2 &S1, S2 \\\cline{1-4}
G\+I\+M\+B\+A\+L &Gimbal control &N/\+A &S1, S2 \\\cline{1-4}
Y6 &Y6-\/copter &M1-\/\+M6 &None \\\cline{1-4}
H\+E\+X6 &Hexacopter-\/\+Plus &M1-\/\+M6 &None \\\cline{1-4}
F\+L\+Y\+I\+N\+G\+\_\+\+W\+I\+N\+G &Fixed wing; elevons &M1 &S1, S2 \\\cline{1-4}
Y4 &Y4-\/copter &M1-\/\+M4 &None \\\cline{1-4}
H\+E\+X6\+X &Hexacopter-\/\+X &M1-\/\+M6 &None \\\cline{1-4}
O\+C\+T\+O\+X8 &Octocopter-\/\+X (over/under) &M1-\/\+M8 &None \\\cline{1-4}
O\+C\+T\+O\+F\+L\+A\+T\+P &Octocopter-\/\+Flat\+Plus &M1-\/\+M8 &None \\\cline{1-4}
O\+C\+T\+O\+F\+L\+A\+T\+X &Octocopter-\/\+Flat\+X &M1-\/\+M8 &None \\\cline{1-4}
A\+I\+R\+P\+L\+A\+N\+E &Fixed wing; Ax2, R, E &M1 &S1, S2, S3, S4 \\\cline{1-4}
H\+E\+L\+I\+\_\+120\+\_\+\+C\+C\+P\+M &&&\\\cline{1-4}
H\+E\+L\+I\+\_\+90\+\_\+\+D\+E\+G &&&\\\cline{1-4}
V\+T\+A\+I\+L4 &Quadcopter with V-\/\+Tail &M1-\/\+M4 &N/\+A \\\cline{1-4}
H\+E\+X6\+H &Hexacopter-\/\+H &M1-\/\+M6 &None \\\cline{1-4}
P\+P\+M\+\_\+\+T\+O\+\_\+\+S\+E\+R\+V\+O &&&\\\cline{1-4}
D\+U\+A\+L\+C\+O\+P\+T\+E\+R &Dualcopter &M1-\/\+M2 &S1, S2 \\\cline{1-4}
S\+I\+N\+G\+L\+E\+C\+O\+P\+T\+E\+R &Conventional helicopter &M1 &S1 \\\cline{1-4}
A\+T\+A\+I\+L4 &Quadcopter with A-\/\+Tail &M1-\/\+M4 &N/\+A \\\cline{1-4}
C\+U\+S\+T\+O\+M &User-\/defined &&\\\cline{1-4}
C\+U\+S\+T\+O\+M A\+I\+R\+P\+L\+A\+N\+E &User-\/defined airplane &&\\\cline{1-4}
C\+U\+S\+T\+O\+M T\+R\+I\+C\+O\+P\+T\+E\+R &User-\/defined tricopter &&\\\cline{1-4}
\end{TabularC}
\subsection*{Servo configuration}

The cli {\ttfamily servo} command defines the settings for the servo outputs. The cli mixer {\ttfamily smix} command controllers how the mixer maps internal F\+C data (R\+C input, P\+I\+D stabilisation output, channel forwarding, etc) to servo outputs.

\subsection*{Servo filtering}

A low-\/pass filter can be enabled for the servos. It may be useful for avoiding structural modes in the airframe, for example.

\subsubsection*{Configuration}

Currently it can only be configured via the C\+L\+I\+:


\begin{DoxyEnumerate}
\item Use {\ttfamily set servo\+\_\+lowpass\+\_\+freq = nnn} to select the cutoff frequency. Valid values range from 10 to 400. This is a fraction of the loop frequency in 1/1000ths. For example, {\ttfamily 40} means {\ttfamily 0.\+040}.
\item Use {\ttfamily set servo\+\_\+lowpass\+\_\+enable = 1} to enable filtering.
\end{DoxyEnumerate}

The cutoff frequency can be determined by the following formula\+: {\ttfamily Frequency = 1000 $\ast$ servo\+\_\+lowpass\+\_\+freq / looptime}

For example, if {\ttfamily servo\+\_\+lowpass\+\_\+freq} is set to 40, and looptime is set to the default of 3500 us, the cutoff frequency will be 11.\+43 Hz.

\subsubsection*{Tuning}

One method for tuning the filter cutoff is as follows\+:


\begin{DoxyEnumerate}
\item Ensure your vehicle can move at least somewhat freely in the troublesome axis. For example, if you are having yaw oscillations on a tricopter, ensure that the copter is supported in a way that allows it to rotate left and right to at least some degree. Suspension near the C\+G is ideal. Alternatively, you can just fly the vehicle and trigger the problematic condition you are trying to eliminate, although tuning will be more tedious.
\item Tap the vehicle at its end in the axis under evaluation. Directly commanding the servo in question to move may also be used. In the tricopter example, tap the end of the tail boom from the side, or command a yaw using your transmitter.
\item If your vehicle oscillates for several seconds or even continues oscillating indefinitely, then the filter cutoff frequency should be reduced. Reduce the value of {\ttfamily servo\+\_\+lowpass\+\_\+freq} by half its current value and repeat the previous step.
\item If the oscillations are dampened within roughly a second or are no longer present, then you are done. Be sure to run {\ttfamily save}.
\end{DoxyEnumerate}

\subsection*{Custom Motor Mixing}

Custom motor mixing allows for completely customized motor configurations. Each motor must be defined with a custom mixing table for that motor. The mix must reflect how close each motor is with reference to the C\+G (Center of Gravity) of the flight controller. A motor closer to the C\+G of the flight controller will need to travel less distance than a motor further away.

Steps to configure custom mixer in the C\+L\+I\+:


\begin{DoxyEnumerate}
\item Use {\ttfamily mixer custom} to enable the custom mixing.
\item Use {\ttfamily mmix reset} to erase the any existing custom mixing.
\item Issue a {\ttfamily mmix} statement for each motor.
\end{DoxyEnumerate}

The mmix statement has the following syntax\+: {\ttfamily mmix n T\+H\+R\+O\+T\+T\+L\+E R\+O\+L\+L P\+I\+T\+C\+H Y\+A\+W}

\begin{TabularC}{2}
\hline
\rowcolor{lightgray}{\bf Mixing table parameter }&{\bf Definition  }\\\cline{1-2}
n &Motor ordering number \\\cline{1-2}
T\+H\+R\+O\+T\+T\+L\+E &All motors that are used in this configuration are set to 1.\+0. Unused set to 0.\+0. \\\cline{1-2}
R\+O\+L\+L &Indicates how much roll authority this motor imparts to the roll of the flight controller. Accepts values nominally from 1.\+0 to -\/1.\+0. \\\cline{1-2}
P\+I\+T\+C\+H &Indicates the pitch authority this motor has over the flight controller. Also accepts values nominally from 1.\+0 to -\/1.\+0. \\\cline{1-2}
Y\+A\+W &Indicates the direction of the motor rotation in relationship with the flight controller. 1.\+0 = C\+C\+W -\/1.\+0 = C\+W. \\\cline{1-2}
\end{TabularC}
Note\+: the {\ttfamily mmix} command may show a motor mix that is not active, custom motor mixes are only active for models that use custom mixers.

Note\+: You have to configure every motor number starting at 0. Your command will be ignored if there was no {\ttfamily mmix} command for the previous motor number. See example 5.

\subsection*{Custom Servo Mixing}

Custom servo mixing rules can be applied to each servo. Rules are applied in the order they are defined.

\begin{TabularC}{2}
\hline
\rowcolor{lightgray}{\bf id }&{\bf Servo slot  }\\\cline{1-2}
0 &G\+I\+M\+B\+A\+L P\+I\+T\+C\+H \\\cline{1-2}
1 &G\+I\+M\+B\+A\+L R\+O\+L\+L \\\cline{1-2}
2 &E\+L\+E\+V\+A\+T\+O\+R / S\+I\+N\+G\+L\+E\+C\+O\+P\+T\+E\+R\+\_\+4 \\\cline{1-2}
3 &F\+L\+A\+P\+P\+E\+R\+O\+N 1 (L\+E\+F\+T) / S\+I\+N\+G\+L\+E\+C\+O\+P\+T\+E\+R\+\_\+1 \\\cline{1-2}
4 &F\+L\+A\+P\+P\+E\+R\+O\+N 2 (R\+I\+G\+H\+T) / B\+I\+C\+O\+P\+T\+E\+R\+\_\+\+L\+E\+F\+T / D\+U\+A\+L\+C\+O\+P\+T\+E\+R\+\_\+\+L\+E\+F\+T / S\+I\+N\+G\+L\+E\+C\+O\+P\+T\+E\+R\+\_\+2 \\\cline{1-2}
5 &R\+U\+D\+D\+E\+R / B\+I\+C\+O\+P\+T\+E\+R\+\_\+\+R\+I\+G\+H\+T / D\+U\+A\+L\+C\+O\+P\+T\+E\+R\+\_\+\+R\+I\+G\+H\+T / S\+I\+N\+G\+L\+E\+C\+O\+P\+T\+E\+R\+\_\+3 \\\cline{1-2}
6 &T\+H\+R\+O\+T\+T\+L\+E (Based O\+N\+L\+Y on the first motor output) \\\cline{1-2}
7 &F\+L\+A\+P\+S \\\cline{1-2}
\end{TabularC}


\begin{TabularC}{2}
\hline
\rowcolor{lightgray}{\bf id }&{\bf Input sources  }\\\cline{1-2}
0 &Stabilised R\+O\+L\+L \\\cline{1-2}
1 &Stabilised P\+I\+T\+C\+H \\\cline{1-2}
2 &Stabilised Y\+A\+W \\\cline{1-2}
3 &Stabilised T\+H\+R\+O\+T\+T\+L\+E \\\cline{1-2}
4 &R\+C R\+O\+L\+L \\\cline{1-2}
5 &R\+C P\+I\+T\+C\+H \\\cline{1-2}
6 &R\+C Y\+A\+W \\\cline{1-2}
7 &R\+C T\+H\+R\+O\+T\+T\+L\+E \\\cline{1-2}
8 &R\+C A\+U\+X 1 \\\cline{1-2}
9 &R\+C A\+U\+X 2 \\\cline{1-2}
10 &R\+C A\+U\+X 3 \\\cline{1-2}
11 &R\+C A\+U\+X 4 \\\cline{1-2}
12 &G\+I\+M\+B\+A\+L P\+I\+T\+C\+H \\\cline{1-2}
13 &G\+I\+M\+B\+A\+L R\+O\+L\+L \\\cline{1-2}
\end{TabularC}
Note\+: the {\ttfamily smix} command may show a servo mix that is not active, custom servo mixes are only active for models that use custom mixers.

\subsection*{Servo Reversing}

Servos are reversed using the {\ttfamily smix reverse} command.

e.\+g. when using the T\+R\+I mixer to reverse the tail servo on a tricopter use this\+:

{\ttfamily smix reverse 5 2 r}

i.\+e. when mixing rudder servo slot ({\ttfamily 5}) using Stabilised Y\+A\+W input source ({\ttfamily 2}) reverse the direction ({\ttfamily r})

{\ttfamily smix reverse} is a per-\/profile setting. So ensure you configure it for your profiles as required.

\subsubsection*{Example 1\+: A K\+K2.\+0 wired motor setup}

Here's an example of a X configuration quad, but the motors are still wired using the K\+K board motor numbering scheme.

``` K\+K2.\+0 Motor Layout

1\+C\+W 2\+C\+C\+W \textbackslash{} / K\+K / \textbackslash{} 4\+C\+C\+W 3\+C\+W ```


\begin{DoxyEnumerate}
\item Use {\ttfamily mixer custom}
\item Use {\ttfamily mmix reset}
\item Use {\ttfamily mmix 0 1.\+0, 1.\+0, -\/1.\+0, -\/1.\+0} for the Front Left motor. It tells the flight controller the \#1 motor is used, provides positive roll, provides negative pitch and is turning C\+W.
\item Use {\ttfamily mmix 1 1.\+0, -\/1.\+0, -\/1.\+0, 1.\+0} for the Front Right motor. It still provides a negative pitch authority, but unlike the front left, it provides negative roll authority and turns C\+C\+W.
\item Use {\ttfamily mmix 2 1.\+0, -\/1.\+0, 1.\+0, -\/1.\+0} for the Rear Right motor. It has negative roll, provides positive pitch when the speed is increased and turns C\+W.
\item Use {\ttfamily mmix 3 1.\+0, 1.\+0, 1.\+0, 1.\+0} for the Rear Left motor. Increasing motor speed imparts positive roll, positive pitch and turns C\+C\+W.
\end{DoxyEnumerate}

\subsubsection*{Example 2\+: A H\+E\+X-\/\+U Copter}

Here is an example of a U-\/shaped hex; probably good for herding giraffes in the Sahara. Because the 1 and 6 motors are closer to the roll axis, they impart much less force than the motors mounted twice as far from the F\+C C\+G. The effect they have on pitch is the same as the forward motors because they are the same distance from the F\+C C\+G. The 2 and 5 motors do not contribute anything to pitch because speeding them up and slowing them down has no effect on the forward/back pitch of the F\+C.

``` H\+E\+X6-\/\+U

.4........3. ............ .5...F\+C...2. ............ ...6....1... ```

\begin{TabularC}{4}
\hline
\rowcolor{lightgray}{\bf Command}&{\bf Roll }&{\bf Pitch }&{\bf Yaw  }\\\cline{1-4}
Use {\ttfamily mmix 0 1.\+0, -\/0.\+5, 1.\+0, -\/1.\+0} &half negative &full positive &C\+W \\\cline{1-4}
Use {\ttfamily mmix 1 1.\+0, -\/1.\+0, 0.\+0, 1.\+0} &full negative &none &C\+C\+W \\\cline{1-4}
Use {\ttfamily mmix 2 1.\+0, -\/1.\+0, -\/1.\+0, -\/1.\+0} &full negative &full negative &C\+W \\\cline{1-4}
Use {\ttfamily mmix 3 1.\+0, 1.\+0, -\/1.\+0, 1.\+0} &full positive &full negative &C\+C\+W \\\cline{1-4}
Use {\ttfamily mmix 4 1.\+0, 1.\+0, 0.\+0, -\/1.\+0} &full positive &none &C\+W \\\cline{1-4}
Use {\ttfamily mmix 5 1.\+0, 0.\+5, 1.\+0, 1.\+0} &half positive &full positive &C\+C\+W \\\cline{1-4}
\end{TabularC}
\subsubsection*{Example 3\+: Custom tricopter}

``` mixer C\+U\+S\+T\+O\+M\+T\+R\+I mmix reset mmix 0 1.\+000 0.\+000 1.\+333 0.\+000 mmix 1 1.\+000 -\/1.\+000 -\/0.\+667 0.\+000 mmix 2 1.\+000 1.\+000 -\/0.\+667 0.\+000 smix reset smix 0 5 2 100 0 0 100 0 profile 0 smix reverse 5 2 r profile 1 smix reverse 5 2 r profile 2 smix reverse 5 2 r

```

\subsubsection*{Example 4\+: Custom Airplane with Differential Thrust}

Here is an example of a custom twin engine plane with \href{http://rcvehicles.about.com/od/rcairplanes/ss/RCAirplaneBasic.htm#step8}{\tt Differential Thrust} Motors take the first 2 pins, the servos take pins as indicated in the \mbox{[}Servo slot\mbox{]} chart above. Settings bellow have motor yaw influence at \char`\"{}0.\+3\char`\"{}, you can change this nuber to have more or less differential thrust over the two motors. Note\+: You can look at the Motors tab in \href{https://chrome.google.com/webstore/detail/cleanflight-configurator/enacoimjcgeinfnnnpajinjgmkahmfgb?hl=en}{\tt Cleanflight Cofigurator} to see motor and servo outputs.

\begin{TabularC}{2}
\hline
\rowcolor{lightgray}{\bf Pins }&{\bf Outputs  }\\\cline{1-2}
1 &Left Engine \\\cline{1-2}
2 &Right Engine \\\cline{1-2}
3 &\mbox{[}E\+M\+P\+T\+Y\mbox{]} \\\cline{1-2}
4 &Roll / Aileron \\\cline{1-2}
5 &Roll / Aileron \\\cline{1-2}
6 &Yaw / Rudder \\\cline{1-2}
7 &Pitch / Elevator \\\cline{1-2}
8 &\mbox{[}E\+M\+P\+T\+Y\mbox{]} \\\cline{1-2}
\end{TabularC}
``` mixer C\+U\+S\+T\+O\+M\+A\+I\+R\+P\+L\+A\+N\+E mmix reset mmix 0 1.\+0 0.\+0 0.\+0 0.\+3 \# Left Engine mmix 1 1.\+0 0.\+0 0.\+0 -\/0.\+3 \# Right Engine

smix reset \section*{Rule Servo Source Rate Speed Min Max Box}

smix 0 3 0 100 0 0 100 0 \# Roll / Aileron smix 1 4 0 100 0 0 100 0 \# Roll / Aileron smix 3 5 2 100 0 0 100 0 \# Yaw / Rudder smix 2 2 1 100 0 0 100 0 \# Pitch / Elevator

```

\subsubsection*{Example 5\+: Use motor output 0,1,2,4 because your output 3 is broken}

For this to work you have to make a dummy mmix for motor 3. We do this by just saying it has 0 impact on yaw,roll and pitch. ``` mixer custom mmix reset mmix 0 1.\+0, -\/1.\+0, 1.\+0, -\/1.\+0 mmix 1 1.\+0, -\/1.\+0, -\/1.\+0, 1.\+0 mmix 2 1.\+0, 1.\+0, 1.\+0, 1.\+0 mmix 3 1.\+0, 0.\+0, 0.\+0, 0.\+0 mmix 4 1.\+0, 1.\+0, -\/1.\+0, -\/1.\+0 save ``` 