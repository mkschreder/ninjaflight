Cleanflight supports the use of addressable L\+E\+D strips. Addressable L\+E\+D strips allow each L\+E\+D in the strip to be programmed with a unique and independant color. This is far more advanced than the normal R\+G\+B strips which require that all the L\+E\+Ds in the strip show the same color.

Addressable L\+E\+D strips can be used to show information from the flight controller system, the current implementation supports the following\+:


\begin{DoxyItemize}
\item Up to 32 L\+E\+Ds.
\item Indicators showing pitch/roll stick positions.
\item Heading/\+Orientation lights.
\item Flight mode specific color schemes.
\item Low battery warning.
\item A\+U\+X operated on/off switch.
\item G\+P\+S state.
\item R\+S\+S\+I level.
\end{DoxyItemize}

Support for more than 32 L\+E\+Ds is possible, it just requires additional development.

\subsection*{Supported hardware}

Only strips of 32 W\+S2811/\+W\+S2812 L\+E\+Ds are supported currently. If the strip is longer than 32 L\+E\+Ds it does not matter, but only the first 32 are used.

W\+S2812 L\+E\+Ds require an 800khz signal and precise timings and thus requires the use of a dedicated hardware timer.

Note\+: Not all W\+S2812 I\+Cs use the same timings, some batches use different timings.

It could be possible to be able to specify the timings required via C\+L\+I if users request it.

\subsubsection*{Tested Hardware}


\begin{DoxyItemize}
\item \href{https://www.adafruit.com/products/2226}{\tt Adafruit Neo\+Pixel Jewel 7} (preliminary testing)
\begin{DoxyItemize}
\item Measured current consumption in all white mode $\sim$ 350 m\+A.
\item Fits well under motors on mini 250 quads.
\end{DoxyItemize}
\item \href{https://www.adafruit.com/products/1426}{\tt Adafruit Neo\+Pixel Stick} (works well)
\begin{DoxyItemize}
\item Measured current consumption in all white mode $\sim$ 350 m\+A.
\end{DoxyItemize}
\end{DoxyItemize}

\subsection*{Connections}

W\+S2812 L\+E\+D strips generally require a single data line, 5\+V and G\+N\+D.

W\+S2812 L\+E\+Ds on full brightness can consume quite a bit of current. It is recommended to verify the current draw and ensure your supply can cope with the load. On a multirotor that uses multiple B\+E\+C E\+S\+C's you can try use a different B\+E\+C to the one the F\+C uses. e.\+g. E\+S\+C1/\+B\+E\+C1 -\/$>$ F\+C, E\+S\+C2/\+B\+E\+C2 -\/$>$ L\+E\+D strip. It's also possible to power one half of the strip from one B\+E\+C and the other half from another B\+E\+C. Just ensure that the G\+R\+O\+U\+N\+D is the same for all B\+E\+C outputs and L\+E\+Ds.

\begin{TabularC}{4}
\hline
\rowcolor{lightgray}{\bf Target }&{\bf Pin }&{\bf L\+E\+D Strip }&{\bf Signal  }\\\cline{1-4}
Naze/\+Olimexino &R\+C5 &Data In &P\+A6 \\\cline{1-4}
C\+C3\+D &R\+C\+O5 &Data In &P\+B4 \\\cline{1-4}
Chebuzz\+F3/\+F3\+Discovery &P\+B8 &Data In &P\+B8 \\\cline{1-4}
Sparky &P\+W\+M5 &Data In &P\+A6 \\\cline{1-4}
\end{TabularC}
Since R\+C5 is also used for Soft\+Serial on the Naze/\+Olimexino it means that you cannot use Soft\+Serial and led strips at the same time. Additionally, since R\+C5 is also used for Parallel P\+W\+M R\+C input on both the Naze, Chebuzz and S\+T\+M32\+F3\+Discovery targets, led strips can not be used at the same time at Parallel P\+W\+M.

If you have L\+E\+Ds that are intermittent, flicker or show the wrong colors then drop the V\+I\+N to less than 4.\+7v, e.\+g. by using an inline diode on the V\+I\+N to the L\+E\+D strip. The problem occurs because of the difference in voltage between the data signal and the power signal. The W\+S2811 L\+E\+D's require the data signal (Din) to be between 0.\+3 $\ast$ Vin (Max) and 0.\+7 $\ast$ V\+I\+N (Min) to register valid logic low/high signals. The L\+E\+D pin on the C\+P\+U will always be between 0v to $\sim$3.3v, so the Vin should be 4.\+7v (3.\+3v / 0.\+7 = 4.\+71v). Some L\+E\+Ds are more tolerant of this than others.

The datasheet can be found here\+: \href{http://www.adafruit.com/datasheets/WS2812.pdf}{\tt http\+://www.\+adafruit.\+com/datasheets/\+W\+S2812.\+pdf}

\subsection*{Configuration}

The led strip feature can be configured via the G\+U\+I.

G\+U\+I\+: Enable the Led Strip feature via the G\+U\+I under setup.

Configure the leds from the Led Strip tab in the cleanflight G\+U\+I. First setup how the led's are laid out so that you can visualize it later as you configure and so the flight controller knows how many led's there are available.

There is a step by step guide on how to use the G\+U\+I to configure the Led Strip feature using the G\+U\+I \href{http://blog.oscarliang.net/setup-rgb-led-cleanflight/}{\tt http\+://blog.\+oscarliang.\+net/setup-\/rgb-\/led-\/cleanflight/} which was published early 2015 by Oscar Liang which may or may not be up-\/to-\/date by the time you read this.

C\+L\+I\+: Enable the {\ttfamily L\+E\+D\+\_\+\+S\+T\+R\+I\+P} feature via the cli\+:

``` feature L\+E\+D\+\_\+\+S\+T\+R\+I\+P ```

If you enable L\+E\+D\+\_\+\+S\+T\+R\+I\+P feature and the feature is turned off again after a reboot then check your config does not conflict with other features, as above.

Configure the L\+E\+Ds using the {\ttfamily led} command.

The {\ttfamily led} command takes either zero or two arguments -\/ an zero-\/based led number and a sequence which indicates pair of coordinates, direction flags and mode flags and a color.

If used with zero arguments it prints out the led configuration which can be copied for future reference.

Each led is configured using the following template\+: {\ttfamily x,y\+:ddd\+:mmm\+:cc}

{\ttfamily x} and {\ttfamily y} are grid coordinates of a 0 based 16x16 grid, north west is 0,0, south east is 15,15 {\ttfamily ddd} specifies the directions, since an led can face in any direction it can have multiple directions. Directions are\+:

{\ttfamily N} -\/ North {\ttfamily E} -\/ East {\ttfamily S} -\/ South {\ttfamily W} -\/ West {\ttfamily U} -\/ Up {\ttfamily D} -\/ Down

For instance, an L\+E\+D that faces South-\/east at a 45 degree downwards angle could be configured as {\ttfamily S\+E\+D}.

Note\+: It is perfectly possible to configure an L\+E\+D to have all directions {\ttfamily N\+E\+S\+W\+U\+D} but probably doesn't make sense.

{\ttfamily mmm} specifies the modes that should be applied an L\+E\+D. Modes are\+:


\begin{DoxyItemize}
\item {\ttfamily W} -\/ {\ttfamily W}warnings.
\item {\ttfamily F} -\/ {\ttfamily F}light mode \& Orientation
\item {\ttfamily I} -\/ {\ttfamily I}ndicator.
\item {\ttfamily A} -\/ {\ttfamily A}rmed state.
\item {\ttfamily T} -\/ {\ttfamily T}hrust state.
\item {\ttfamily R} -\/ {\ttfamily R}ing thrust state.
\item {\ttfamily C} -\/ {\ttfamily C}olor.
\item {\ttfamily G} -\/ {\ttfamily G}P\+S state.
\item {\ttfamily S} -\/ R{\ttfamily S}S\+S\+I level.
\item {\ttfamily B} -\/ {\ttfamily B}link (flash twice) mode.
\end{DoxyItemize}

{\ttfamily cc} specifies the color number (0 based index).

Example\+:

``` led 0 0,15\+:S\+D\+:\+I\+A\+W\+:0 led 1 15,0\+:N\+D\+:\+I\+A\+W\+:0 led 2 0,0\+:N\+D\+:\+I\+A\+W\+:0 led 3 0,15\+:S\+D\+:\+I\+A\+W\+:0 led 4 7,7\+::\+C\+:1 led 5 8,8\+::\+C\+:2 led 6 8,9\+::\+B\+:1 ```

To erase an led, and to mark the end of the chain, use {\ttfamily 0,0\+:\+:} as the second argument, like this\+:

``` led 4 0,0\+:\+:\+: ```

It is best to erase all L\+E\+Ds that you do not have connected.

\subsubsection*{Modes}

\paragraph*{Warning}

This mode simply uses the L\+E\+Ds to flash when warnings occur.

\begin{TabularC}{3}
\hline
\rowcolor{lightgray}{\bf Warning }&{\bf L\+E\+D Pattern }&{\bf Notes  }\\\cline{1-3}
Arm-\/lock enabled &flash between green and off &occurs calibration or when unarmed and the aircraft is tilted too much \\\cline{1-3}
Low Battery &flash red and off &battery monitoring must be enabled. May trigger temporarily under high-\/throttle due to voltage drop \\\cline{1-3}
Failsafe &flash between light blue and yellow &Failsafe must be enabled \\\cline{1-3}
\end{TabularC}
Flash patterns appear in order, so that it's clear which warnings are enabled.

\paragraph*{G\+P\+S state}

This mode shows the G\+P\+S state and satellite count.

No fix = red L\+E\+D 3\+D fix = green L\+E\+D

The L\+E\+Ds will blink as many times as the satellite count, then pause and start again.

\paragraph*{R\+S\+S\+I level}

This mode fades the L\+E\+D current L\+E\+D color to the previous/next color in the H\+S\+B color space depending on R\+S\+S\+I level. When the R\+S\+S\+I level is at the mean value the color is unaffected, thus it can be mixed with orientation colors to indicate orientation and R\+S\+S\+I at the same time. R\+S\+S\+I should normally be combined with Color or Mode/\+Orientation.

\paragraph*{Blink}

This mode blinks the current L\+E\+D, alternatively from black to the selected color.

\paragraph*{Flight Mode \& Orientation}

This mode shows the flight mode and orientation.

When flight modes are active then the L\+E\+Ds are updated to show different colors depending on the mode, placement on the grid and direction.

L\+E\+Ds are set in a specific order\+:
\begin{DoxyItemize}
\item L\+E\+Ds that marked as facing up or down.
\item L\+E\+Ds that marked as facing west or east A\+N\+D are on the west or east side of the grid.
\item L\+E\+Ds that marked as facing north or south A\+N\+D are on the north or south side of the grid.
\end{DoxyItemize}

That is, south facing L\+E\+Ds have priority.

The mapping between modes led placement and colors is currently fixed and cannot be changed.

\paragraph*{Indicator}

This mode flashes L\+E\+Ds that correspond to roll and pitch stick positions. i.\+e. they indicate the direction the craft is going to turn.

\begin{TabularC}{3}
\hline
\rowcolor{lightgray}{\bf Mode }&{\bf Direction }&{\bf L\+E\+D Color  }\\\cline{1-3}
Orientation &North &W\+H\+I\+T\+E \\\cline{1-3}
Orientation &East &D\+A\+R\+K V\+I\+O\+L\+E\+T \\\cline{1-3}
Orientation &South &R\+E\+D \\\cline{1-3}
Orientation &West &D\+E\+E\+P P\+I\+N\+K \\\cline{1-3}
Orientation &Up &B\+L\+U\+E \\\cline{1-3}
Orientation &Down &O\+R\+A\+N\+G\+E \\\cline{1-3}
&&\\\cline{1-3}
Head Free &North &L\+I\+M\+E G\+R\+E\+E\+N \\\cline{1-3}
Head Free &East &D\+A\+R\+K V\+I\+O\+L\+E\+T \\\cline{1-3}
Head Free &South &O\+R\+A\+N\+G\+E \\\cline{1-3}
Head Free &West &D\+E\+E\+P P\+I\+N\+K \\\cline{1-3}
Head Free &Up &B\+L\+U\+E \\\cline{1-3}
Head Free &Down &O\+R\+A\+N\+G\+E \\\cline{1-3}
&&\\\cline{1-3}
Horizon &North &B\+L\+U\+E \\\cline{1-3}
Horizon &East &D\+A\+R\+K V\+I\+O\+L\+E\+T \\\cline{1-3}
Horizon &South &Y\+E\+L\+L\+O\+W \\\cline{1-3}
Horizon &West &D\+E\+E\+P P\+I\+N\+K \\\cline{1-3}
Horizon &Up &B\+L\+U\+E \\\cline{1-3}
Horizon &Down &O\+R\+A\+N\+G\+E \\\cline{1-3}
&&\\\cline{1-3}
Angle &North &C\+Y\+A\+N \\\cline{1-3}
Angle &East &D\+A\+R\+K V\+I\+O\+L\+E\+T \\\cline{1-3}
Angle &South &Y\+E\+L\+L\+O\+W \\\cline{1-3}
Angle &West &D\+E\+E\+P P\+I\+N\+K \\\cline{1-3}
Angle &Up &B\+L\+U\+E \\\cline{1-3}
Angle &Down &O\+R\+A\+N\+G\+E \\\cline{1-3}
&&\\\cline{1-3}
Mag &North &M\+I\+N\+T G\+R\+E\+E\+N \\\cline{1-3}
Mag &East &D\+A\+R\+K V\+I\+O\+L\+E\+T \\\cline{1-3}
Mag &South &O\+R\+A\+N\+G\+E \\\cline{1-3}
Mag &West &D\+E\+E\+P P\+I\+N\+K \\\cline{1-3}
Mag &Up &B\+L\+U\+E \\\cline{1-3}
Mag &Down &O\+R\+A\+N\+G\+E \\\cline{1-3}
&&\\\cline{1-3}
Baro &North &L\+I\+G\+H\+T B\+L\+U\+E \\\cline{1-3}
Baro &East &D\+A\+R\+K V\+I\+O\+L\+E\+T \\\cline{1-3}
Baro &South &R\+E\+D \\\cline{1-3}
Baro &West &D\+E\+E\+P P\+I\+N\+K \\\cline{1-3}
Baro &Up &B\+L\+U\+E \\\cline{1-3}
Baro &Down &O\+R\+A\+N\+G\+E \\\cline{1-3}
\end{TabularC}
\paragraph*{Armed state}

This mode toggles L\+E\+Ds between green and blue when disarmed and armed, respectively.

Note\+: Armed State cannot be used with Flight Mode.

\paragraph*{Thrust state}

This mode fades the L\+E\+D current L\+E\+D color to the previous/next color in the H\+S\+B color space depending on throttle stick position. When the throttle is in the middle position the color is unaffected, thus it can be mixed with orientation colors to indicate orientation and throttle at the same time. Thrust should normally be combined with Color or Mode/\+Orientation.

\paragraph*{Thrust ring state}

This mode is allows you to use one or multiple led rings (e.\+g. Neo\+Pixel ring) for an afterburner effect. The light pattern rotates clockwise as throttle increases.

A better effect is acheived when L\+E\+Ds configured for thrust ring have no other functions.

L\+E\+D direction and X/\+Y positions are irrelevant for thrust ring L\+E\+D state. The order of the L\+E\+Ds that have the state determines how the L\+E\+D behaves.

Each L\+E\+D of the ring can be a different color. The color can be selected between the 16 colors availables.

For example, led 0 is set as a {\ttfamily R}ing thrust state led in color 13 as follow.

``` led 0 2,2\+::\+R\+:13 ```

L\+E\+D strips and rings can be combined.

\paragraph*{Solid Color}

The mode allows you to set an L\+E\+D to be permanently on and set to a specific color.

x,y position and directions are ignored when using this mode.

Other modes will override or combine with the color mode.

For example, to set led 0 to always use color 10 you would issue this command.

``` led 0 0,0\+::\+C\+:10 ```

\subsubsection*{Colors}

Colors can be configured using the cli {\ttfamily color} command.

The {\ttfamily color} command takes either zero or two arguments -\/ an zero-\/based color number and a sequence which indicates pair of hue, saturation and value (H\+S\+V).

See \href{http://en.wikipedia.org/wiki/HSL_and_HSV}{\tt http\+://en.\+wikipedia.\+org/wiki/\+H\+S\+L\+\_\+and\+\_\+\+H\+S\+V}

If used with zero arguments it prints out the color configuration which can be copied for future reference.

The default color configuration is as follows\+:

\begin{TabularC}{2}
\hline
\rowcolor{lightgray}{\bf Index }&{\bf Color  }\\\cline{1-2}
0 &black \\\cline{1-2}
1 &white \\\cline{1-2}
2 &red \\\cline{1-2}
3 &orange \\\cline{1-2}
4 &yellow \\\cline{1-2}
5 &lime green \\\cline{1-2}
6 &green \\\cline{1-2}
7 &mint green \\\cline{1-2}
8 &cyan \\\cline{1-2}
9 &light blue \\\cline{1-2}
10 &blue \\\cline{1-2}
11 &dark violet \\\cline{1-2}
12 &magenta \\\cline{1-2}
13 &deep pink \\\cline{1-2}
14 &black \\\cline{1-2}
15 &black \\\cline{1-2}
\end{TabularC}
``` color 0 0,0,0 color 1 0,255,255 color 2 0,0,255 color 3 30,0,255 color 4 60,0,255 color 5 90,0,255 color 6 120,0,255 color 7 150,0,255 color 8 180,0,255 color 9 210,0,255 color 10 240,0,255 color 11 270,0,255 color 12 300,0,255 color 13 330,0,255 color 14 0,0,0 color 15 0,0,0 ```

\subsubsection*{Mode Colors Assignement}

Mode Colors can be configured using the cli {\ttfamily mode\+\_\+color} command.


\begin{DoxyItemize}
\item No arguments\+: lists all mode colors
\item arguments\+: mode, function, color
\end{DoxyItemize}

First 6 groups of Mode\+Indexes are \+:

\begin{TabularC}{2}
\hline
\rowcolor{lightgray}{\bf mode }&{\bf name  }\\\cline{1-2}
0 &orientation \\\cline{1-2}
1 &headfree \\\cline{1-2}
2 &horizon \\\cline{1-2}
3 &angle \\\cline{1-2}
4 &mag \\\cline{1-2}
5 &baro \\\cline{1-2}
6 &special \\\cline{1-2}
\end{TabularC}
Modes 0 to 5 functions\+:

\begin{TabularC}{2}
\hline
\rowcolor{lightgray}{\bf function }&{\bf name  }\\\cline{1-2}
0 &north \\\cline{1-2}
1 &east \\\cline{1-2}
2 &south \\\cline{1-2}
3 &west \\\cline{1-2}
4 &up \\\cline{1-2}
5 &down \\\cline{1-2}
\end{TabularC}
Mode 6 use these functions\+:

\begin{TabularC}{2}
\hline
\rowcolor{lightgray}{\bf function }&{\bf name  }\\\cline{1-2}
0 &disarmed \\\cline{1-2}
1 &armed \\\cline{1-2}
2 &animation \\\cline{1-2}
3 &background \\\cline{1-2}
4 &blink background \\\cline{1-2}
5 &gps\+: no satellites \\\cline{1-2}
6 &gps\+: no fix \\\cline{1-2}
7 &gps\+: 3\+D fix \\\cline{1-2}
\end{TabularC}
The Color\+Index is picked from the colors array (\char`\"{}palette\char`\"{}).

Examples (using the default colors)\+:


\begin{DoxyItemize}
\item set armed color to red\+: {\ttfamily mode\+\_\+color 6 1 2}
\item set disarmed color to yellow\+: {\ttfamily mode\+\_\+color 6 0 4}
\item set Headfree mode 'south' to Cyan\+: {\ttfamily mode\+\_\+color 1 2 8}
\end{DoxyItemize}

\subsection*{Positioning}

Cut the strip into sections as per diagrams below. When the strips are cut ensure you reconnect each output to each input with cable where the break is made. e.\+g. connect 5\+V out to 5\+V in, G\+N\+D to G\+N\+D and Data Out to Data In.

Orientation is when viewed with the front of the aircraft facing away from you and viewed from above.

\subsubsection*{Example 12 L\+E\+D config}

The default configuration is as follows ``` led 0 15,15\+:E\+S\+:\+I\+A\+:0 led 1 15,8\+:E\+:\+W\+F\+:0 led 2 15,7\+:E\+:\+W\+F\+:0 led 3 15,0\+:N\+E\+:\+I\+A\+:0 led 4 8,0\+:N\+:\+F\+:0 led 5 7,0\+:N\+:\+F\+:0 led 6 0,0\+:N\+W\+:\+I\+A\+:0 led 7 0,7\+:W\+:\+W\+F\+:0 led 8 0,8\+:W\+:\+W\+F\+:0 led 9 0,15\+:S\+W\+:\+I\+A\+:0 led 10 7,15\+:S\+:\+W\+F\+:0 led 11 8,15\+:S\+:\+W\+F\+:0 led 12 7,7\+:U\+:\+W\+F\+:0 led 13 8,7\+:U\+:\+W\+F\+:0 led 14 7,8\+:D\+:\+W\+F\+:0 led 15 8,8\+:D\+:\+W\+F\+:0 led 16 8,9\+::\+R\+:3 led 17 9,10\+::\+R\+:3 led 18 10,11\+::\+R\+:3 led 19 10,12\+::\+R\+:3 led 20 9,13\+::\+R\+:3 led 21 8,14\+::\+R\+:3 led 22 7,14\+::\+R\+:3 led 23 6,13\+::\+R\+:3 led 24 5,12\+::\+R\+:3 led 25 5,11\+::\+R\+:3 led 26 6,10\+::\+R\+:3 led 27 7,9\+::\+R\+:3 led 28 0,0\+:\+:\+:0 led 29 0,0\+:\+:\+:0 led 30 0,0\+:\+:\+:0 led 31 0,0\+:\+:\+:0 ```

Which translates into the following positions\+:

``` 6 3 \textbackslash{} / \textbackslash{} 5-\/4 / \textbackslash{} F\+R\+O\+N\+T / 7,8 $\vert$ 12-\/15 $\vert$ 1,2 / B\+A\+C\+K \textbackslash{} / 10,11 \textbackslash{} / \textbackslash{} 9 0 R\+I\+N\+G 16-\/27 ```

L\+E\+Ds 0,3,6 and 9 should be placed underneath the quad, facing downwards. L\+E\+Ds 1-\/2, 4-\/5, 7-\/8 and 10-\/11 should be positioned so the face east/north/west/south, respectively. L\+E\+Ds 12-\/13 should be placed facing down, in the middle L\+E\+Ds 14-\/15 should be placed facing up, in the middle L\+E\+Ds 16-\/17 should be placed in a ring and positioned at the rear facing south.

This is the default so that if you don't want to place L\+E\+Ds top and bottom in the middle just connect the first 12 L\+E\+Ds.

\subsubsection*{Example 16 L\+E\+D config}

``` led 0 15,15\+:S\+D\+:\+I\+A\+:0 led 1 8,8\+:E\+:\+F\+W\+:0 led 2 8,7\+:E\+:\+F\+W\+:0 led 3 15,0\+:N\+D\+:\+I\+A\+:0 led 4 7,7\+:N\+:\+F\+W\+:0 led 5 8,7\+:N\+:\+F\+W\+:0 led 6 0,0\+:N\+D\+:\+I\+A\+:0 led 7 7,7\+:W\+:\+F\+W\+:0 led 8 7,8\+:W\+:\+F\+W\+:0 led 9 0,15\+:S\+D\+:\+I\+A\+:0 led 10 7,8\+:S\+:\+F\+W\+:0 led 11 8,8\+:S\+:\+F\+W\+:0 led 12 7,7\+:D\+:\+F\+W\+:0 led 13 8,7\+:D\+:\+F\+W\+:0 led 14 7,7\+:U\+:\+F\+W\+:0 led 15 8,7\+:U\+:\+F\+W\+:0 ```

Which translates into the following positions\+:

``` 6 3 \textbackslash{} / \textbackslash{} 5-\/4 / 7 \textbackslash{} F\+R\+O\+N\+T / 2 $\vert$ 12-\/15 $\vert$ 8 / B\+A\+C\+K \textbackslash{} 1 / 10-\/11 \textbackslash{} / \textbackslash{} 9 0 ```

L\+E\+Ds 0,3,6 and 9 should be placed underneath the quad, facing downwards. L\+E\+Ds 1-\/2, 4-\/5, 7-\/8 and 10-\/11 should be positioned so the face east/north/west/south, respectively. L\+E\+Ds 12-\/13 should be placed facing down, in the middle L\+E\+Ds 14-\/15 should be placed facing up, in the middle

\subsubsection*{Exmple 28 L\+E\+D config}

``` \#right rear cluster led 0 9,9\+:S\+:\+F\+W\+T\+:0 led 1 10,10\+:S\+:\+F\+W\+T\+:0 led 2 11,11\+:S\+:\+I\+A\+:0 led 3 11,11\+:E\+:\+I\+A\+:0 led 4 10,10\+:E\+:\+A\+T\+:0 led 5 9,9\+:E\+:\+A\+T\+:0 \section*{right front cluster}

led 6 10,5\+:S\+:\+F\+:0 led 7 11,4\+:S\+:\+F\+:0 led 8 12,3\+:S\+:\+I\+A\+:0 led 9 12,2\+:N\+:\+I\+A\+:0 led 10 11,1\+:N\+:\+F\+:0 led 11 10,0\+:N\+:\+F\+:0 \section*{center front cluster}

led 12 7,0\+:N\+:\+F\+W\+:0 led 13 6,0\+:N\+:\+F\+W\+:0 led 14 5,0\+:N\+:\+F\+W\+:0 led 15 4,0\+:N\+:\+F\+W\+:0 \section*{left front cluster}

led 16 2,0\+:N\+:\+F\+:0 led 17 1,1\+:N\+:\+F\+:0 led 18 0,2\+:N\+:\+I\+A\+:0 led 19 0,3\+:W\+:\+I\+A\+:0 led 20 1,4\+:S\+:\+F\+:0 led 21 2,5\+:S\+:\+F\+:0 \section*{left rear cluster}

led 22 2,9\+:W\+:\+A\+T\+:0 led 23 1,10\+:W\+:\+A\+T\+:0 led 24 0,11\+:W\+:\+I\+A\+:0 led 25 0,11\+:S\+:\+I\+A\+:0 led 26 1,10\+:S\+:\+F\+W\+T\+:0 led 27 2,9\+:S\+:\+F\+W\+T\+:0 ```

``` 16-\/18 9-\/11 19-\/21 \textbackslash{} / 6-\/8 \textbackslash{} 12-\/15 / \textbackslash{} F\+R\+O\+N\+T / / B\+A\+C\+K \textbackslash{} / \textbackslash{} 22-\/24 / \textbackslash{} 3-\/5 25-\/27 0-\/2 ```

All L\+E\+Ds should face outwards from the chassis in this configuration.

Note\+: This configuration is specifically designed for the \href{http://www.goodluckbuy.com/alien-spider-aq50d-pro-250mm-mini-quadcopter-carbon-fiber-micro-multicopter-frame.html}{\tt Alien Spider A\+Q50\+D P\+R\+O 250mm frame}.

\subsection*{Troubleshooting}

On initial power up the L\+E\+Ds on the strip will be set to W\+H\+I\+T\+E. This means you can attach a current meter to verify the current draw if your measurement equipment is fast enough. Most 5050 L\+E\+Ds will draw 0.\+3 Watts a piece. This also means that you can make sure that each R,G and B L\+E\+D in each L\+E\+D module on the strip is also functioning.

After a short delay the L\+E\+Ds will show the unarmed color sequence and or low-\/battery warning sequence.

Also check that the feature {\ttfamily L\+E\+D\+\_\+\+S\+T\+R\+I\+P} was correctly enabled and that it does not conflict with other features, as above. 