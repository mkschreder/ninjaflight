Some newer boards with full U\+S\+B support must be flashed in U\+S\+B D\+F\+U mode. This is a straightforward process in Configurator versions 0.\+67 and newer. The standard flashing procedure should work successfully with the caveat of some platform specific problems as noted below. The \char`\"{}\+No reboot sequence\char`\"{} checkbox has no effect as the device will automatically be detected when already in bootloader mode (a D\+F\+U device will appear in the connect dropdown if this is the case). The Full chip erase checkbox operates as normal. The baudrate checkbox is ignored as it has no relevance to U\+S\+B.

\subsection*{Platform Specific\+: Linux}

Linux requires udev rules to allow write access to U\+S\+B devices for users. An example shell command to acheive this on Ubuntu is shown here\+: ``` (echo '\# D\+F\+U (Internal bootloader for S\+T\+M32 M\+C\+Us)' echo 'S\+U\+B\+S\+Y\+S\+T\+E\+M==\char`\"{}usb\char`\"{}, A\+T\+T\+R\+S\{id\+Vendor\}==\char`\"{}0483\char`\"{}, A\+T\+T\+R\+S\{id\+Product\}==\char`\"{}df11\char`\"{}, M\+O\+D\+E=\char`\"{}0664\char`\"{}, G\+R\+O\+U\+P=\char`\"{}plugdev\char`\"{}') $\vert$ sudo tee /etc/udev/rules.d/45-\/stdfu-\/permissions.\+rules $>$ /dev/null ```

This assigns the device to the plugdev group(a standard group in Ubuntu). To check that your account is in the plugdev group type {\ttfamily groups} in the shell and ensure plugdev is listed. If not you can add yourself as shown (replacing {\ttfamily $<$username$>$} with your username)\+: ``` sudo usermod -\/a -\/\+G plugdev $<$username$>$ ```

If you see your tty\+U\+S\+B device disappear right after the board is connected, chances are that the Modem\+Manager service (that handles network connectivity for you) thinks it is a G\+S\+M modem. If this happens, you can issue the following command to disable the service\+: ``` sudo systemctl stop Modem\+Manager.\+service ```

If your system lacks the systemctl command, use any equivalent command that works on your system to disable services. You can likely add your device I\+D to a blacklist configuration file to stop Modem\+Manager from touching the device, if you need it for cellural networking, but that is beyond the scope of cleanflight documentation.

If you see the tty\+U\+S\+B device appear and immediately disappear from the list in Cleanflight Configurator when you plug in your flight controller via U\+S\+B, chances are that Network\+Manager thinks your board is a G\+S\+M modem and hands it off to the Modem\+Manager daemon as the flight controllers are not known to the blacklisted

\subsection*{Platform Specific\+: Windows}

Chrome can have problems accessing U\+S\+B devices on Windows. A driver should be automatically installed by Windows for the S\+T Device in D\+F\+U Mode but this doesn't always allow access for Chrome. The solution is to replace the S\+T driver with a libusb driver. The easiest way to do that is to download \href{http://zadig.akeo.ie/}{\tt Zadig}. With the board connected and in bootloader mode (reset it by sending the character R via serial, or simply attempt to flash it with the correct serial port selected in Configurator)\+:
\begin{DoxyItemize}
\item Open Zadig
\item Choose Options $>$ List All Devices
\item Select {\ttfamily S\+T\+M32 B\+O\+O\+T\+L\+O\+A\+D\+E\+R} in the device list
\item Choose {\ttfamily Win\+U\+S\+B (v6.\+x.\+x.\+x)} in the right hand box 
\item Click Replace Driver
\item Restart Chrome (make sure it is completely closed, logout and login if unsure)
\item Now the D\+F\+U device should be seen by Configurator 
\end{DoxyItemize}