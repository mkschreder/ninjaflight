Cleanflight supports a buzzer which is used for the following purposes\+:


\begin{DoxyItemize}
\item Low and critical battery alarms (when battery monitoring enabled)
\item Arm/disarm tones (and warning beeps while armed)
\item Notification of calibration complete status
\item T\+X-\/\+A\+U\+X operated beeping -\/ useful for locating your aircraft after a crash
\item Failsafe status
\item Flight mode change
\item Rate profile change (via T\+X-\/\+A\+U\+X switch)
\end{DoxyItemize}

If the arm/disarm is via the control stick, holding the stick in the disarm position will sound a repeating tone. This can be used as a lost-\/model locator.

Three beeps immediately after powering the board means that the gyroscope calibration has completed successfully. Cleanflight calibrates the gyro automatically upon every power-\/up. It is important that the copter stay still on the ground until the three beeps sound, so that gyro calibration isn't thrown off. If you move the copter significantly during calibration, Cleanflight will detect this, and will automatically re-\/start the calibration once the copter is still again. This will delay the \char`\"{}three beeps\char`\"{} tone. If you move the copter just a little bit, the gyro calibration may be incorrect, and the copter may not fly correctly. In this case, the gyro calibration can be performed manually via \hyperlink{Controls_8md}{stick command}, or you may simply power cycle the board.

There is a special arming tone used if a G\+P\+S fix has been attained, and there's a \char`\"{}ready\char`\"{} tone sounded after a G\+P\+S fix has been attained (only happens once). The tone sounded via the T\+X-\/\+A\+U\+X-\/switch will count out the number of satellites (if G\+P\+S fix).

The C\+L\+I command {\ttfamily play\+\_\+sound} is useful for demonstrating the buzzer tones. Repeatedly entering the command will play the various tones in turn. Entering the command with a numeric-\/index parameter (see below) will play the associated tone.

Buzzer is enabled by default on platforms that have buzzer connections.

\subsection*{Tone sequences}

Buzzer tone sequences (square wave generation) are made so that \+: 1st, 3rd, 5th, .. are the delays how long the beeper is on and 2nd, 4th, 6th, .. are the delays how long beeper is off. Delays are in milliseconds/10 (i.\+e., 5 =$>$ 50ms).

Sequences available in Cleanflight v1.\+9 and above are \+: \begin{DoxyVerb}0    GYRO_CALIBRATED       20, 10, 20, 10, 20, 10   Gyro is calibrated
1    RX_LOST_LANDING       10, 10, 10, 10, 10, 40, 40, 10, 40, 10, 40, 40, 10, 10, 10, 10, 10, 70    SOS morse code
2    RX_LOST               50, 50       TX off or signal lost (repeats until TX is okay)
3    DISARMING             15, 5, 15, 5     Disarming the board
4    ARMING                30, 5, 5, 5      Arming the board
5    ARMING_GPS_FIX        5, 5, 15, 5, 5, 5, 15, 30    Arming and GPS has fix
6    BAT_CRIT_LOW          50, 2        Battery is critically low (repeats)
7    BAT_LOW               25, 50       Battery is getting low (repeats) 
8    NULL                  multi beeps      GPS status (sat count)
9    RX_SET                10, 10       RX is set (when aux channel is set for beep or beep sequence how many satellites has found if GPS enabled)
10   ACC_CALIBRATION       5, 5, 5, 5       ACC inflight calibration completed
11   ACC_CALIBRATION_FAIL  20, 15, 35, 5    ACC inflight calibration failed
12   READY_BEEP            4, 5, 4, 5, 8, 5, 15, 5, 8, 5, 4, 5, 4, 5    GPS locked and copter ready   
13   NULL                  multi beeps      Variable # of beeps (confirmation, GPS sat count, etc)
14   DISARM_REPEAT         0, 100, 10       Stick held in disarm position (after pause)
15   ARMED                 0, 245, 10, 5    Board is armed (after pause ; repeats until board is disarmed or throttle is increased)
\end{DoxyVerb}


\subsection*{Types of buzzer supported}

The buzzers are enabled/disabled by simply enabling or disabling a G\+P\+I\+O output pin on the board. This means the buzzer must be able to generate its own tone simply by having power applied to it.

Buzzers that need an analog or P\+W\+M signal do not work and will make clicking noises or no sound at all.

Examples of a known-\/working buzzers.


\begin{DoxyItemize}
\item \href{http://www.rapidonline.com/Audio-Visual/Hcm1205x-Miniature-Buzzer-5v-35-0055}{\tt Hcm1205x Miniature Buzzer 5v}
\item \href{http://www.banggood.com/10Pcs-5V-Electromagnetic-Active-Buzzer-Continuous-Beep-Continuously-p-943524.html}{\tt 5\+V Electromagnetic Active Buzzer Continuous Beep}
\item \href{http://www.radioshack.com/pc-board-12vdc-70db-piezo-buzzer/2730074.html#.VIAtpzHF_Si}{\tt Radio Shack Model\+: 273-\/074 P\+C-\/\+B\+O\+A\+R\+D 12\+V\+D\+C (3-\/16v) 70\+D\+B P\+I\+E\+Z\+O B\+U\+Z\+Z\+E\+R}
\item \href{http://uk.farnell.com/multicomp/mckpx-g1205a-3700/transducer-thru-hole-4v-30ma/dp/2135914?CMP=i-bf9f-00001000}{\tt Multi\+Comp M\+C\+K\+P\+X-\/\+G1205\+A-\/3700 T\+R\+A\+N\+S\+D\+U\+C\+E\+R, T\+H\+R\+U-\/\+H\+O\+L\+E, 4\+V, 30\+M\+A}
\item \href{http://www.banggood.com/3-24V-Piezo-Electronic-Tone-Buzzer-Alarm-95DB-Continuous-Sound-p-919348.html}{\tt 3-\/24\+V Piezo Electronic Tone Buzzer Alarm 95\+D\+B}
\end{DoxyItemize}

\subsection*{Connections}

\subsubsection*{Naze32}

Connect a supported buzzer directly to the B\+U\+Z\+Z pins. Observe polarity. Also if you are working with flight controller outside of a craft, on a bench for example, you need to supply 5 volts and ground to one of the E\+S\+C connections or the buzzer will not function.

\subsubsection*{C\+C3\+D}

Buzzer support on the C\+C3\+D requires that a buzzer circuit be created to which the input is P\+A15. P\+A15 is unused and not connected according to the C\+C3\+D Revision A schematic. Connecting to P\+A15 requires careful soldering.

See the \href{Wiring/CC3D - buzzer circuit.pdf}{\tt C\+C3\+D -\/ buzzer circuit.\+pdf} for details. 