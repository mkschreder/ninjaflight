Build a binary with debugging information using command line or via Eclipse make target.

Example Eclipse make target



\section*{G\+D\+B and Open\+O\+C\+D}

start openocd

Create a new debug configuration in eclipse \+:  

you can control openocd with a telnet connection\+: \begin{DoxyVerb} telnet localhost 4444
\end{DoxyVerb}


stop the board, flash the firmware, restart\+: \begin{DoxyVerb} reset halt
 wait_halt 
 sleep 100
 poll
 flash probe 0
 flash write_image erase /home/user/git/cleanflight/obj/cleanflight_NAZE.hex 0x08000000
 sleep 200
 soft_reset_halt
 wait_halt
 poll
 reset halt
\end{DoxyVerb}


A this point you can launch the debug in Eclispe. 

\section*{G\+D\+B and J Link}

Here are some screenshots showing Hydra's configuration of Eclipse (Kepler)

If you use cygwin to build the binaries then be sure to have configured your common {\ttfamily Source Lookup Path}, {\ttfamily Path Mappings} first, like this\+:



Create a new {\ttfamily G\+D\+B Hardware Debugging} launch configuration from the {\ttfamily Run} menu

It's important to have build the executable compiled with G\+D\+B debugging information first. Select the appropriate .elf file (not hex file) -\/ In these examples the target platform is an O\+L\+I\+M\+E\+X\+I\+N\+O, not a naze32.

D\+I\+S\+A\+B\+L\+E auto-\/build



Choose the appropriate gdb executable -\/ ideally from the same toolchain that you use to build the executable.



Configure Startup as follows

Initialization commands

``` target remote localhost\+:2331 monitor interface S\+W\+D monitor speed 2000 monitor flash device = S\+T\+M32\+F103\+R\+B monitor flash download = 1 monitor flash breakpoints = 1 monitor endian little monitor reset ```





It may be useful to specify run commands too\+:

``` monitor reg r13 = (0x00000000) monitor reg pc = (0x00000004) continue ```



If you use cygwin an additional entry should be shown on the Source tab (not present in this screenshot)



Nothing to change from the defaults on the Common tab



Start up the J-\/\+Link server in U\+S\+B mode



If it connects to your target device it should look like this



From Eclipse launch the application using the Run/\+Debug Configurations..., Eclipse should upload the compiled file to the target device which looks like this



When it's running the J-\/\+Link server should look like this.



Then finally you can use Eclipse debug features to inspect variables, memory, stacktrace, set breakpoints, step over code, etc.



If Eclipse can't find your breakpoints and they are ignored then check your path mappings (if using cygwin) or use the other debugging launcher as follows. Note the 'Select other...' at the bottom of the configuration window.

 