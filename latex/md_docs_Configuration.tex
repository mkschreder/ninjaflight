Cleanflight is configured primarily using the Cleanflight Configurator G\+U\+I.

Both the command line interface and G\+U\+I are accessible by connecting to a serial port on the target, be it a U\+S\+B virtual serial port, physical hardware U\+A\+R\+T port or a Soft\+Serial port.

See the Serial section for more information and see the Board specific sections for details of the serial ports available on the board you are using.

The G\+U\+I cannot currently configure all aspects of the system, the C\+L\+I must be used to enable or configure some features and settings.

{\bfseries Due to ongoing development, the fact that the G\+U\+I cannot yet backup all your settings and automatic chrome updates of the G\+U\+I app it is highly advisable to backup your settings (using the C\+L\+I) so that when a new version of the configurator or firmware is released you can re-\/apply your settings.}

\subsection*{G\+U\+I}



The G\+U\+I tool is the preferred way of configuration. The G\+U\+I tool also includes a terminal which can be used to interact with the C\+L\+I.

\href{https://chrome.google.com/webstore/detail/cleanflight-configurator/enacoimjcgeinfnnnpajinjgmkahmfgb}{\tt Cleanflight Configurator on Chrome store}

If you cannot use the latest version of the G\+U\+I to access the F\+C due to firmware compatibility issues you can still access the F\+C via the C\+L\+I to backup your settings, or you can install an old version of the configurator.

Old versions of the configurator can be downloaded from the configurator releases page\+: \href{https://github.com/cleanflight/cleanflight-configurator/releases}{\tt https\+://github.\+com/cleanflight/cleanflight-\/configurator/releases} See the R\+E\+A\+D\+M\+E file that comes with the configurator for installation instructions.

\subsection*{C\+L\+I}

Cleanflight can also be configured by a command line interface.

See the \hyperlink{Cli_8md}{C\+L\+I section} of the documentation for more details. 