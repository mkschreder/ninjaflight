A receiver is used to receive radio control signals from your transmitter and convert them into signals that the flight controller can understand.

There are 3 basic types of receivers\+:


\begin{DoxyEnumerate}
\item Parallel P\+W\+M Receivers
\item P\+P\+M Receivers
\item Serial Receivers
\end{DoxyEnumerate}

As of 2016 the recommendation for new purchases is a Serial or P\+P\+M based receiver. Avoid Parallel P\+W\+M recievers (1 wire per channel). This is due to the amount of I\+O pins parallel P\+W\+M based receivers use. Some new F\+C's do not support parallel P\+W\+M.

\subsection*{Parallel P\+W\+M Receivers}

8 channel support, 1 channel per input pin. On some platforms using parallel input will disable the use of serial ports and Soft\+Serial making it hard to use telemetry or G\+P\+S features.

\subsection*{P\+P\+M Receivers}

P\+P\+M is sometimes known as P\+P\+M S\+U\+M or C\+P\+P\+M.

12 channels via a single input pin, not as accurate or jitter free as methods that use serial communications, but readily available.

These receivers are reported working\+:


\begin{DoxyItemize}
\item \href{http://www.frsky-rc.com/product/pro.php?pro_id=24}{\tt Fr\+Sky D4\+R-\/\+I\+I}
\item \href{http://www.graupner.de/en/products/33512/product.aspx}{\tt Graupner G\+R24}
\item \href{http://orangerx.com/2014/05/20/r615x-spektrumjr-dsm2dsmx-compatible-6ch-2-4ghz-receiver-wcppm-2/}{\tt R615\+X Spektrum/\+J\+R D\+S\+M2/\+D\+S\+M\+X Compatible 6\+Ch 2.\+4\+G\+Hz Receiver w/\+C\+P\+P\+M}
\item \href{http://www.frsky-rc.com/product/pro.php?pro_id=21}{\tt Fr\+Sky D8\+R-\/\+X\+P 8ch telemetry receiver, or C\+P\+P\+M and R\+S\+S\+I enabled receiver}
\item \href{http://www.frsky-rc.com/download/view.php?sort=&down=158&file=X4R-X4RSB}{\tt Fr\+Sky X4\+R and Fr\+Sky X4\+R\+S\+B} when flashed with C\+P\+P\+M firmware and bound with jumper between signal pins 2 and 3
\item All Fr\+Sky S.\+Bus enabled devices when connected with \href{http://www.frsky-rc.com/product/pro.php?pro_id=112}{\tt S.\+Bus C\+P\+P\+M converter cable}. Without jumper this converter cable uses 21ms frame size (Channels 1-\/8). When jumper is in place, it uses 28ms frame and channels 1-\/12 are available
\item Fly\+Sky/\+Turnigy F\+S-\/i\+A6\+B receiver for F\+S-\/i6 and F\+S-\/i10 transmitters
\end{DoxyItemize}

\subsection*{Serial Receivers}

\subsubsection*{Spektrum}

8 channels via serial currently supported.

These receivers are reported working\+:

Lemon Rx D\+S\+M\+X Compatible P\+P\+M 8-\/\+Channel Receiver + Lemon D\+S\+M\+X Compatible Satellite with Failsafe \href{http://www.lemon-rx.com/shop/index.php?route=product/product&product_id=118}{\tt http\+://www.\+lemon-\/rx.\+com/shop/index.\+php?route=product/product\&product\+\_\+id=118}

\subsubsection*{S.\+B\+U\+S}

16 channels via serial currently supported. See below how to set up your transmitter.


\begin{DoxyItemize}
\item You probably need an inverter between the receiver output and the flight controller. However, some flight controllers have this built in (the main port on C\+C3\+D, for example), and doesn't need one.
\item Some Open\+L\+R\+S receivers produce a non-\/inverted S\+B\+U\+S signal. It is possible to switch S\+B\+U\+S inversion off using C\+L\+I command {\ttfamily set sbus\+\_\+inversion = O\+F\+F} when using an F3 based flight controller.
\item Softserial ports cannot be used with S\+B\+U\+S because it runs at too high of a bitrate (1\+Mbps). Refer to the chapter specific to your board to determine which port(s) may be used.
\item You will need to configure the channel mapping in the G\+U\+I (Receiver tab) or C\+L\+I ({\ttfamily map} command). Note that channels above 8 are mapped \char`\"{}straight\char`\"{}, with no remapping.
\end{DoxyItemize}

These receivers are reported working\+:

Fr\+Sky X4\+R\+S\+B 3/16ch Telemetry Receiver \href{http://www.frsky-rc.com/product/pro.php?pro_id=135}{\tt http\+://www.\+frsky-\/rc.\+com/product/pro.\+php?pro\+\_\+id=135}

Fr\+Sky X8\+R 8/16ch Telemetry Receiver \href{http://www.frsky-rc.com/product/pro.php?pro_id=105}{\tt http\+://www.\+frsky-\/rc.\+com/product/pro.\+php?pro\+\_\+id=105}

Futaba R2008\+S\+B 2.\+4\+G\+Hz S-\/\+F\+H\+S\+S \href{http://www.futaba-rc.com/systems/futk8100-8j/}{\tt http\+://www.\+futaba-\/rc.\+com/systems/futk8100-\/8j/}

\paragraph*{Open\+T\+X S.\+B\+U\+S configuration}

If using Open\+T\+X set the transmitter module to D16 mode and A\+L\+S\+O select C\+H1-\/16 on the transmitter before binding to allow reception of all 16 channels.

Open\+T\+X 2.\+09, which is shipped on some Taranis X9\+D Plus transmitters, has a bug -\/ \href{https://github.com/opentx/opentx/issues/1701}{\tt issue\+:1701}. The bug prevents use of all 16 channels. Upgrade to the latest Open\+T\+X version to allow correct reception of all 16 channels, without the fix you are limited to 8 channels regardless of the C\+H1-\/16/\+D16 settings.

\subsubsection*{X\+B\+U\+S}

The firmware currently supports the M\+O\+D\+E B version of the X\+Bus protocol. Make sure to set your T\+X to use \char`\"{}\+M\+O\+D\+E B\char`\"{} for X\+B\+U\+S in the T\+X menus! See here for info on J\+R's X\+B\+U\+S protocol\+: \href{http://www.jrpropo.com/english/propo/XBus/}{\tt http\+://www.\+jrpropo.\+com/english/propo/\+X\+Bus/}

These receivers are reported working\+:

X\+G14 14ch D\+M\+S\+S System w/\+R\+G731\+B\+X X\+Bus Receiver \href{http://www.jramericas.com/233794/JRP00631/}{\tt http\+://www.\+jramericas.\+com/233794/\+J\+R\+P00631/}

There exist a remote receiver made for small B\+N\+F-\/models like the Align T-\/\+Rex 150 helicopter. The code also supports using the Align D\+M\+S\+S R\+J01 receiver directly with the cleanflight software. To use this receiver you must power it with 3\+V from the hardware, and then connect the serial line as other serial R\+X receivers. In order for this receiver to work, you need to specify the X\+B\+U\+S\+\_\+\+M\+O\+D\+E\+\_\+\+B\+\_\+\+R\+J01 for serialrx\+\_\+provider. Note that you need to set your radio mode for X\+B\+U\+S \char`\"{}\+M\+O\+D\+E B\char`\"{} also for this receiver to work. Receiver name\+: Align D\+M\+S\+S R\+J01 (H\+E\+R15001)

\subsubsection*{S\+U\+M\+D}

16 channels via serial currently supported.

These receivers are reported working\+:

G\+R-\/24 receiver Ho\+T\+T \href{http://www.graupner.de/en/products/33512/product.aspx}{\tt http\+://www.\+graupner.\+de/en/products/33512/product.\+aspx}

Graupner receiver G\+R-\/12\+S\+H+ Ho\+T\+T \href{http://www.graupner.de/en/products/870ade17-ace8-427f-943b-657040579906/33565/product.aspx}{\tt http\+://www.\+graupner.\+de/en/products/870ade17-\/ace8-\/427f-\/943b-\/657040579906/33565/product.\+aspx}

\subsubsection*{S\+U\+M\+H}

8 channels via serial currently supported.

S\+U\+M\+H is a legacy Graupner protocol. Graupner have issued a firmware updates for many recivers that lets them use S\+U\+M\+D instead.

\subsubsection*{I\+B\+U\+S}

10 channels via serial currently supported.

I\+B\+U\+S is the Fly\+Sky digital serial protocol and is available with the F\+S-\/\+I\+A6\+B and F\+S-\/\+I\+A10 receivers. The Turnigy T\+G\+Y-\/\+I\+A6\+B and T\+G\+Y-\/\+I\+A10 are the same devices with a different label, therefore they also work.

If you are using a 6ch tx such as the F\+S-\/\+I6 or T\+G\+Y-\/\+I6 then you must flash a 10ch firmware on the tx to make use of these extra channels.

These receivers are reported working\+:

Fly\+Sky/\+Turnigy F\+S-\/i\+A6\+B 6-\/\+Channel Receiver \href{http://www.flysky-cn.com/products_detail/&productId=51.html}{\tt http\+://www.\+flysky-\/cn.\+com/products\+\_\+detail/\&product\+Id=51.\+html}

\subsection*{Multi\+Wii serial protocol (M\+S\+P)}

Allows you to use M\+S\+P commands as the R\+C input. Only 8 channel support to maintain compatibility with M\+S\+P.

\subsection*{Configuration}

There are 3 features that control receiver mode\+:

``` R\+X\+\_\+\+P\+P\+M R\+X\+\_\+\+S\+E\+R\+I\+A\+L R\+X\+\_\+\+P\+A\+R\+A\+L\+L\+E\+L\+\_\+\+P\+W\+M R\+X\+\_\+\+M\+S\+P ```

Only one receiver feature can be enabled at a time.

\subsubsection*{R\+X signal-\/loss detection}

The software has signal loss detection which is always enabled. Signal loss detection is used for safety and failsafe reasons.

The {\ttfamily rx\+\_\+min\+\_\+usec} and {\ttfamily rx\+\_\+max\+\_\+usec} settings helps detect when your R\+X stops sending any data, enters failsafe mode or when the R\+X looses signal.

By default, when the signal loss is detected the F\+C will set pitch/roll/yaw to the value configured for {\ttfamily mid\+\_\+rc}. The throttle will be set to the value configured for {\ttfamily rx\+\_\+min\+\_\+usec} or {\ttfamily mid\+\_\+rc} if using 3\+D feature.

Signal loss can be detected when\+:


\begin{DoxyEnumerate}
\item no rx data is received (due to radio reception, recevier configuration or cabling issues).
\item using Serial R\+X and receiver indicates failsafe condition.
\item using any of the first 4 stick channels do not have a value in the range specified by {\ttfamily rx\+\_\+min\+\_\+usec} and {\ttfamily rx\+\_\+max\+\_\+usec}.
\end{DoxyEnumerate}

\subsubsection*{R\+X loss configuration}

The {\ttfamily rxfail} cli command is used to configure per-\/channel rx-\/loss behaviour. You can use the {\ttfamily rxfail} command to change this behaviour. A flight channel can either be A\+U\+T\+O\+M\+A\+T\+I\+C or H\+O\+L\+D, an A\+U\+X channel can either be S\+E\+T or H\+O\+L\+D.


\begin{DoxyItemize}
\item A\+U\+T\+O\+M\+A\+T\+I\+C -\/ Flight channels are set to safe values (low throttle, mid position for yaw/pitch/roll).
\item H\+O\+L\+D -\/ Channel holds the last value.
\item S\+E\+T -\/ Channel is set to a specific configured value.
\end{DoxyItemize}

The default mode is A\+U\+T\+O\+M\+A\+T\+I\+C for flight channels and H\+O\+L\+D for A\+U\+X channels.

The rxfail command can be used in conjunction with mode ranges to trigger various actions.

The {\ttfamily rxfail} command takes 2 or 3 arguments.
\begin{DoxyItemize}
\item Index of channel (See below)
\item Mode ('a' = A\+U\+T\+O\+M\+A\+T\+I\+C, 'h' = H\+O\+L\+D, 's' = S\+E\+T)
\item A value to use when in S\+E\+T mode.
\end{DoxyItemize}

Channels are always specified in the same order, regardless of your channel mapping.


\begin{DoxyItemize}
\item Roll is 0
\item Pitch is 1
\item Yaw is 2
\item Throttle is 3.
\item Aux channels are 4 onwards.
\end{DoxyItemize}

Examples\+:

To make Throttle channel have an automatic value when R\+X loss is detected\+:

{\ttfamily rxfail 3 a}

To make A\+U\+X4 have a value of 2000 when R\+X loss is detected\+:

{\ttfamily rxfail 7 s 2000}

To make A\+U\+X8 hold it's value when R\+X loss is detected\+:

{\ttfamily rxfail 11 h}

W\+A\+R\+N\+I\+N\+G\+: Always make sure you test the behavior is as expected after configuring rxfail settings!

\paragraph*{{\ttfamily rx\+\_\+min\+\_\+usec}}

The lowest channel value considered valid. e.\+g. P\+W\+M/\+P\+P\+M pulse length

\paragraph*{{\ttfamily rx\+\_\+max\+\_\+usec}}

The highest channel value considered valid. e.\+g. P\+W\+M/\+P\+P\+M pulse length

\subsubsection*{Serial R\+X}

See the Serial chapter for some some R\+X configuration examples.

To setup spectrum on the Naze32 or clones in the G\+U\+I\+:
\begin{DoxyEnumerate}
\item Start on the \char`\"{}\+Ports\char`\"{} tab make sure that U\+A\+R\+T2 has serial R\+X. If not set the checkbox, save and reboot.
\item Move to the \char`\"{}\+Configuration\char`\"{} page and in the upper lefthand corner choose Serial R\+X as the receiver type.
\item Below that choose the type of serial receiver that you are using. Save and reboot.
\end{DoxyEnumerate}

Using C\+L\+I\+: For Serial R\+X enable {\ttfamily R\+X\+\_\+\+S\+E\+R\+I\+A\+L} and set the {\ttfamily serialrx\+\_\+provider} C\+L\+I setting as follows.

\begin{TabularC}{2}
\hline
\rowcolor{lightgray}{\bf Serial R\+X Provider }&{\bf Value  }\\\cline{1-2}
S\+P\+E\+K\+T\+R\+U\+M1024 &0 \\\cline{1-2}
S\+P\+E\+K\+T\+R\+U\+M2048 &1 \\\cline{1-2}
S\+B\+U\+S &2 \\\cline{1-2}
S\+U\+M\+D &3 \\\cline{1-2}
S\+U\+M\+H &4 \\\cline{1-2}
X\+B\+U\+S\+\_\+\+M\+O\+D\+E\+\_\+\+B &5 \\\cline{1-2}
X\+B\+U\+S\+\_\+\+M\+O\+D\+E\+\_\+\+B\+\_\+\+R\+J01 &6 \\\cline{1-2}
I\+B\+U\+S &7 \\\cline{1-2}
\end{TabularC}
\subsubsection*{P\+P\+M/\+P\+W\+M input filtering.}

Hardware input filtering can be enabled if you are experiencing interference on the signal sent via your P\+W\+M/\+P\+P\+M R\+X.

Use the {\ttfamily input\+\_\+filtering\+\_\+mode} C\+L\+I setting to select a mode.

\begin{TabularC}{2}
\hline
\rowcolor{lightgray}{\bf Value }&{\bf Meaning  }\\\cline{1-2}
0 &Disabled \\\cline{1-2}
1 &Enabled \\\cline{1-2}
\end{TabularC}
\subsection*{Receiver configuration.}

\subsubsection*{Fr\+Sky D4\+R-\/\+I\+I}

Set the R\+X for 'No Pulses'. Turn O\+F\+F T\+X and R\+X, Turn O\+N R\+X. Press and release F/\+S button on R\+X. Turn off R\+X.

\subsubsection*{Graupner G\+R-\/24 P\+W\+M}

Set failsafe on the throttle channel in the receiver settings (via transmitter menu) to a value below {\ttfamily rx\+\_\+min\+\_\+usec} using channel mode F\+A\+I\+L\+S\+A\+F\+E. This is the prefered way, since this is {\itshape much faster} detected by the F\+C then a channel that sends no pulses (O\+F\+F).

{\bfseries N\+O\+T\+E\+:} One or more control channels may be set to O\+F\+F to signal a failsafe condition to the F\+C, all other channels {\itshape must} be set to either H\+O\+L\+D or O\+F\+F. Do {\bfseries N\+O\+T U\+S\+E} the mode indicated with F\+A\+I\+L\+S\+A\+F\+E instead, as this combination is N\+O\+T handled correctly by the F\+C.

\subsection*{Receiver Channel Range Configuration.}

The channels defined in Clean\+Flight are as follows\+:

\begin{TabularC}{2}
\hline
\rowcolor{lightgray}{\bf Channel number }&{\bf Channel name  }\\\cline{1-2}
0 &Roll \\\cline{1-2}
1 &Pitch \\\cline{1-2}
2 &Yaw \\\cline{1-2}
3 &Throttle \\\cline{1-2}
\end{TabularC}
If you have a transmitter/receiver, that output a non-\/standard pulse range (i.\+e. 1070-\/1930 as some Spektrum receivers) you could use rx channel range configuration to map actual range of your transmitter to 1000-\/2000 as expected by Cleanflight.

The low and high value of a channel range are often referred to as 'End-\/points'. e.\+g. 'End-\/point adjustments / E\+P\+A'.

All attempts should be made to configure your transmitter/receiver to use the range 1000-\/2000 {\itshape before} using this feature as you will have less preceise control if it is used.

To do this you should figure out what range your transmitter outputs and use these values for rx range configuration. You can do this in a few simple steps\+:

If you have used rc range configuration previously you should reset it to prevent it from altering rc input. Do so by entering the following command in C\+L\+I\+: ``` rxrange reset save ```

Now reboot your F\+C, connect the configurator, go to the {\ttfamily Receiver} tab move sticks on your transmitter and note min and max values of first 4 channels. Take caution as you can accidentally arm your craft. Best way is to move one channel at a time.

Go to C\+L\+I and set the min and max values with the following command\+: ``` rxrange $<$channel\+\_\+number$>$ $<$min$>$ $<$max$>$ ```

For example, if you have the range 1070-\/1930 for the first channel you should use {\ttfamily rxrange 0 1070 1930} in the C\+L\+I. Be sure to enter the {\ttfamily save} command to save the settings.

After configuring channel ranges use the sub-\/trim on your transmitter to set the middle point of pitch, roll, yaw and throttle.

You can also use rxrange to reverse the direction of an input channel, e.\+g. {\ttfamily rxrange 0 2000 1000}. 