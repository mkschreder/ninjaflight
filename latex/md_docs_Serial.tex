Cleanflight has enhanced serial port flexibility but configuration is slightly more complex as a result.

Cleanflight has the concept of a function (M\+S\+P, G\+P\+S, Serial R\+X, etc) and a port (V\+C\+P, U\+A\+R\+Tx, Soft\+Serial x). Not all functions can be used on all ports due to hardware pin mapping, conflicting features, hardware, and software constraints.

\subsection*{Serial port types}


\begin{DoxyItemize}
\item U\+S\+B Virtual Com Port (V\+C\+P) -\/ U\+S\+B pins on a U\+S\+B port connected directly to the processor without requiring a dedicated U\+S\+B to U\+A\+R\+T adapter. V\+C\+P does not 'use' a physical U\+A\+R\+T port.
\item U\+A\+R\+T -\/ A pair of dedicated hardware transmit and receive pins with signal detection and generation done in hardware.
\item Soft\+Serial -\/ A pair of hardware transmit and receive pins with signal detection and generation done in software.
\end{DoxyItemize}

U\+A\+R\+T is the most efficient in terms of C\+P\+U usage. Soft\+Serial is the least efficient and slowest, Soft\+Serial should only be used for low-\/bandwidth usages, such as telemetry transmission.

U\+A\+R\+T ports are sometimes exposed via on-\/board U\+S\+B to U\+A\+R\+T converters, such as the C\+P2102 as found on the Naze and Flip32 boards. If the flight controller does not have an on-\/board U\+S\+B to U\+A\+R\+T converter and doesn't support V\+C\+P then an external U\+S\+B to U\+A\+R\+T board is required. These are sometimes referred to as F\+T\+D\+I boards. F\+T\+D\+I is just a common manufacturer of a chip (the F\+T232\+R\+L) used on U\+S\+B to U\+A\+R\+T boards.

When selecting a U\+S\+B to U\+A\+R\+T converter choose one that has D\+T\+R exposed as well as a selector for 3.\+3v and 5v since they are more useful.

Examples\+:


\begin{DoxyItemize}
\item \href{http://www.banggood.com/FT232RL-FTDI-USB-To-TTL-Serial-Converter-Adapter-Module-For-Arduino-p-917226.html}{\tt F\+T232\+R\+L F\+T\+D\+I U\+S\+B To T\+T\+L Serial Converter Adapter}
\item \href{http://www.banggood.com/Wholesale-USB-To-TTL-Or-COM-Converter-Module-Buildin-in-CP2102-New-p-27989.html}{\tt U\+S\+B To T\+T\+L / C\+O\+M Converter Module buildin-\/in C\+P2102}
\end{DoxyItemize}

Both Soft\+Serial and U\+A\+R\+T ports can be connected to your computer via U\+S\+B to U\+A\+R\+T converter boards.

\subsection*{Serial Configuration}

Serial port configuration is best done via the configurator.

Configure serial ports first, then enable/disable features that use the ports. To configure Soft\+Serial ports the S\+O\+F\+T\+S\+E\+R\+I\+A\+L feature must be also be enabled.

\subsubsection*{Constraints}

If the configuration is invalid the serial port configuration will reset to its defaults and features may be disabled.


\begin{DoxyItemize}
\item There must always be a port available to use for M\+S\+P/\+C\+L\+I.
\item There is a maximum of 2 M\+S\+P ports.
\item To use a port for a function, the function's corresponding feature must be also be enabled. e.\+g. after configuring a port for G\+P\+S enable the G\+P\+S feature.
\item If Soft\+Serial is used, then all Soft\+Serial ports must use the same baudrate.
\item Softserial is limited to 19200 buad.
\item All telemetry systems except M\+S\+P will ignore any attempts to override the baudrate.
\item M\+S\+P/\+C\+L\+I can be shared with E\+I\+T\+H\+E\+R Blackbox O\+R telemetry. In shared mode blackbox or telemetry will be output only when armed.
\item Smartport telemetry cannot be shared with M\+S\+P.
\item No other serial port sharing combinations are valid.
\item You can use as many different telemetry systems as you like at the same time.
\item You can only use each telemetry system once. e.\+g. Fr\+Sky telemetry cannot be used on two port, but M\+S\+P Telemetry + Fr\+Sky on different ports is fine.
\end{DoxyItemize}

\subsubsection*{Configuration via C\+L\+I}

You can use the C\+L\+I for configuration but the commands are reserved for developers and advanced users.

The {\ttfamily serial} C\+L\+I command takes 6 arguments.


\begin{DoxyEnumerate}
\item Identifier
\end{DoxyEnumerate}
\begin{DoxyEnumerate}
\item Function bitmask (see serial\+Port\+Function\+\_\+e in the source)
\end{DoxyEnumerate}
\begin{DoxyEnumerate}
\item M\+S\+P baud rate
\end{DoxyEnumerate}
\begin{DoxyEnumerate}
\item G\+P\+S baud rate
\end{DoxyEnumerate}
\begin{DoxyEnumerate}
\item Telemetry baud rate (auto baud allowed)
\end{DoxyEnumerate}
\begin{DoxyEnumerate}
\item Blackbox baud rate
\end{DoxyEnumerate}

\subsubsection*{Baud Rates}

The allowable baud rates are as follows\+:

\begin{TabularC}{2}
\hline
\rowcolor{lightgray}{\bf Identifier }&{\bf Baud rate  }\\\cline{1-2}
0 &Auto \\\cline{1-2}
1 &9600 \\\cline{1-2}
2 &19200 \\\cline{1-2}
3 &38400 \\\cline{1-2}
4 &57600 \\\cline{1-2}
5 &115200 \\\cline{1-2}
6 &230400 \\\cline{1-2}
7 &250000 \\\cline{1-2}
\end{TabularC}
