As many can attest, multirotors and R\+C models in general can be very dangerous, particularly on the test bench. Here are some simple golden rules to save you a trip to the local E\+R\+:
\begin{DoxyItemize}
\item {\bfseries N\+E\+V\+E\+R} arm your model with propellers fitted unless you intend to fly!
\item {\bfseries Always} remove your propellers if you are setting up for the first time, flashing firmware, or if in any doubt.
\end{DoxyItemize}

\subsection*{Before Installing}

Please consult the \hyperlink{Cli_8md}{Cli}, \hyperlink{Controls_8md}{Controls}, \hyperlink{Failsafe_8md}{Failsafe} and \hyperlink{Modes_8md}{Modes} pages for further important information.

You are highly advised to use the Receiver tab in the Clean\+Flight Configurator, making sure your Rx channel values are centered at 1500 (1520 for Futaba R\+C) with minimum \& maximums of 1000 and 2000 (respectively) are reached when controls are operated. Failure to configure these ranges properly can create problems, such as inability to arm (because you can't reach the endpoints) or immediate activation of \hyperlink{Failsafe_8md}{failsafe}.

You may have to adjust your channel endpoints and trims/sub-\/trims on your R\+C transmitter to achieve the expected range of 1000 to 2000.

The referenced values for each channel have marked impact on the operation of the flight controller and the different flight modes.

\subsection*{Props Spinning When Armed}

With the default configuration, when the controller is armed, the propellers {\itshape W\+I\+L\+L} begin spinning at low speed. We recommend keeping this setting as it provides a good visual indication the craft is armed.

If you wish to change this behavior, see the M\+O\+T\+O\+R\+\_\+\+S\+T\+O\+P feature in the Configurator and relevant documentation pages. Enabling this feature will stop the props from spinning when armed. 