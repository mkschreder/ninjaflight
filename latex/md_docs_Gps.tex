G\+P\+S features in Cleanflight are experimental. Please share your findings with the developers.

G\+P\+S works best if the G\+P\+S receiver is mounted above and away from other sources of interference.

The compass/mag sensor should be well away from sources of magnetic interference, e.\+g. keep it away from power wires, motors, E\+S\+Cs.

Two G\+P\+S protocols are supported. N\+M\+E\+A text and U\+B\+L\+O\+X binary.

\subsection*{Configuration}

Enable the G\+P\+S from the C\+L\+I as follows\+:


\begin{DoxyEnumerate}
\item \hyperlink{Serial_8md}{configure a serial port to use for G\+P\+S.}
\end{DoxyEnumerate}
\begin{DoxyEnumerate}
\item set your G\+P\+S baud rate
\end{DoxyEnumerate}
\begin{DoxyEnumerate}
\item enable the {\ttfamily feature G\+P\+S}
\end{DoxyEnumerate}
\begin{DoxyEnumerate}
\item set the {\ttfamily gps\+\_\+provider}
\end{DoxyEnumerate}
\begin{DoxyEnumerate}
\item connect your G\+P\+S to the serial port configured for G\+P\+S.
\end{DoxyEnumerate}
\begin{DoxyEnumerate}
\item save and reboot.
\end{DoxyEnumerate}

Note\+: G\+P\+S packet loss has been observed at 115200. Try using 57600 if you experience this.

For the connections step check the Board documentation for pins and port numbers.

\subsubsection*{G\+P\+S Provider}

Set the {\ttfamily gps\+\_\+provider} appropriately, for example {\ttfamily set gps\+\_\+provider=U\+B\+L\+O\+X}

\begin{TabularC}{1}
\hline
\rowcolor{lightgray}{\bf Value  }\\\cline{1-1}
N\+M\+E\+A \\\cline{1-1}
U\+B\+L\+O\+X \\\cline{1-1}
\end{TabularC}
\subsubsection*{G\+P\+S Auto configuration}

When using U\+B\+L\+O\+X it is a good idea to use G\+P\+S auto configuration so your F\+C gets the G\+P\+S messages it needs.

Enable G\+P\+S auto configuration as follows {\ttfamily set gps\+\_\+auto\+\_\+config=O\+N}.

If you are not using G\+P\+S auto configuration then ensure your G\+P\+S receiver sends out the correct messages at the right frequency. See below for manual U\+Blox settings.

\subsubsection*{S\+B\+A\+S}

When using a U\+B\+L\+O\+X G\+P\+S the S\+B\+A\+S mode can be configured using {\ttfamily gps\+\_\+sbas\+\_\+mode}.

The default is A\+U\+T\+O.

\begin{TabularC}{2}
\hline
\rowcolor{lightgray}{\bf Value }&{\bf Region  }\\\cline{1-2}
A\+U\+T\+O &Global \\\cline{1-2}
E\+G\+N\+O\+S &Europe \\\cline{1-2}
W\+A\+A\+S &North America \\\cline{1-2}
M\+S\+A\+S &Asia \\\cline{1-2}
G\+A\+G\+A\+N &India \\\cline{1-2}
\end{TabularC}
If you use a regional specific setting you may achieve a faster G\+P\+S lock than using A\+U\+T\+O.

This setting only works when {\ttfamily gps\+\_\+auto\+\_\+config=O\+N}

\subsection*{G\+P\+S Receiver Configuration}

U\+Blox G\+P\+S units can either be configured using the F\+C or manually.

\subsubsection*{U\+Blox G\+P\+S manual configuration}

Use U\+Box U-\/\+Center and connect your G\+P\+S to your computer. The C\+L\+I {\ttfamily gpspassthrough} command may be of use if you do not have a spare U\+S\+A\+R\+T to U\+S\+B adapter.

Note that many boards will not provide +5\+V from U\+S\+B to the G\+P\+S module, such as the S\+P\+Racing\+F3; if you are using {\ttfamily gpspassthrough} you may need to connect a B\+E\+C to the controller if your board permits it, or use a standalone U\+A\+R\+T adapter. Check your board documentation to see if your G\+P\+S port is powered from U\+S\+B.

Display the Packet Console (so you can see what messages your receiver is sending to your computer).

Display the Configation View.

Navigate to C\+F\+G (Configuration)

Select {\ttfamily Revert to default configuration}. Click {\ttfamily Send}.

At this point you might need to disconnect and reconnect at the default baudrate -\/ probably 9600 baud.

Navigate to P\+R\+T (Ports)

Set {\ttfamily Target} to {\ttfamily 1 -\/ Uart 1} Set {\ttfamily Protocol In} to {\ttfamily 0+1+2} Set {\ttfamily Protocol Out} to {\ttfamily 0+1} Set {\ttfamily Buadrate} to {\ttfamily 57600} {\ttfamily 115200} Press {\ttfamily Send}

This will immediatly \char`\"{}break\char`\"{} communication to the G\+P\+S. Since you haven't saved the new baudrate setting to the non-\/volatile memory you need to change the baudrate you communicate to the G\+P\+S without resetting the G\+P\+S. So {\ttfamily Disconnect}, Change baud rate to match, then {\ttfamily Connect}.

Click on {\ttfamily P\+R\+T} in the Configuration view again and inspect the packet console to make sure messages are being sent and acknowledged.

Next, to ensure the F\+C doesn't waste time processing unneeded messages, click on {\ttfamily M\+S\+G} and enable the following on U\+A\+R\+T1 alone with a rate of 1. When changing message target and rates remember to click {\ttfamily Send} after changing each message.\+: \begin{DoxyVerb}NAV-POSLLH
NAV-DOP
NAV-SOL
NAV-VELNED
NAV-TIMEUTC
\end{DoxyVerb}


Enable the following on U\+A\+R\+T1 with a rate of 5, to reduce bandwidth and load on the F\+C. \begin{DoxyVerb}NAV-SVINFO
\end{DoxyVerb}


All other message types should be disabled.

Next change the global update rate, click {\ttfamily Rate (Rates)} in the Configuration view.

Set {\ttfamily Measurement period} to {\ttfamily 100} ms. Set {\ttfamily Navigation rate} to {\ttfamily 1}. Click {\ttfamily Send}.

This will cause the G\+P\+S receive to send the require messages out 10 times a second. If your G\+P\+S receiver cannot be set to use {\ttfamily 100}ms try {\ttfamily 200}ms (5hz) -\/ this is less precise.

Next change the mode, click {\ttfamily N\+A\+V5 (Navigation 5)} in the Configuration View.

Set to {\ttfamily Dynamic Model} to {\ttfamily Pedestrian} and click {\ttfamily Send}.

Next change the S\+B\+A\+S settings. Click {\ttfamily S\+B\+A\+S (S\+B\+A\+S Settings)} in the Configuration View.

Set {\ttfamily Subsystem} to {\ttfamily Enabled}. Set {\ttfamily P\+R\+N Codes} to {\ttfamily Auto-\/\+Scan}. Click {\ttfamily Send}.

Finally, we need to save the configuration.

Click {\ttfamily C\+F\+G (Configuration} in the Configuration View.

Select {\ttfamily Save current configuration} and click {\ttfamily Send}.

\subsubsection*{U\+Blox Navigation model}

Cleanflight will use {\ttfamily Pedestrian} when gps auto config is used.

From the U\+Blox documentation\+:


\begin{DoxyItemize}
\item Pedestrian -\/ Applications with low acceleration and speed, e.\+g. how a pedestrian would move. Low acceleration assumed. M\+A\+X Altitude \mbox{[}m\mbox{]}\+: 9000, M\+A\+X Velocity \mbox{[}m/s\mbox{]}\+: 30, M\+A\+X Vertical, Velocity \mbox{[}m/s\mbox{]}\+: 20, Sanity check type\+: Altitude and Velocity, Max Position Deviation\+: Small.
\item Portable -\/ Applications with low acceleration, e.\+g. portable devices. Suitable for most situations. M\+A\+X Altitude \mbox{[}m\mbox{]}\+: 12000, M\+A\+X Velocity \mbox{[}m/s\mbox{]}\+: 310, M\+A\+X Vertical Velocity \mbox{[}m/s\mbox{]}\+: 50, Sanity check type\+: Altitude and Velocity, Max Position Deviation\+: Medium.
\item Airborne $<$ 1\+G -\/ Used for applications with a higher dynamic range and vertical acceleration than a passenger car. No 2\+D position fixes supported. M\+A\+X Altitude \mbox{[}m\mbox{]}\+: 50000, M\+A\+X Velocity \mbox{[}m/s\mbox{]}\+: 100, M\+A\+X Vertical Velocity \mbox{[}m/s\mbox{]}\+: 100, Sanity check type\+: Altitude, Max Position Deviation\+: Large
\end{DoxyItemize}

\subsection*{Hardware}

There are many G\+P\+S receivers available on the market. Below are some examples of user-\/tested hardware.

\subsubsection*{Ublox}

\subsubsection*{U-\/\+Blox}

\paragraph*{N\+E\+O-\/\+M8}

\begin{TabularC}{2}
\hline
\rowcolor{lightgray}{\bf Module }&{\bf Comments  }\\\cline{1-2}
U-\/blox Neo-\/\+M8\+N w/\+Compass &Pinout can be found in Pixfalcon manual. S\+D\+A and S\+C\+L can be attached to I2\+C bus for compass, T\+X and R\+X can be attached to U\+A\+R\+T for G\+P\+S. Power must be applied for either to function. \\\cline{1-2}
Reyax R\+Y825\+A\+I &N\+E\+O-\/\+M8\+N, 18\+Hz U\+A\+R\+T U\+S\+B interface G\+P\+S Glonass Bei\+Dou Q\+Z\+S\+S antenna module flash. \href{http://www.ebay.com/itm/RY825AI-18Hz-UART-USB-interface-GPS-Glonass-BeiDou-QZSS-antenna-module-flash/181566850426}{\tt e\+Bay} \\\cline{1-2}
\end{TabularC}
\paragraph*{N\+E\+O-\/7}

\begin{TabularC}{2}
\hline
\rowcolor{lightgray}{\bf Module }&{\bf Comments  }\\\cline{1-2}
U-\/blox Neo-\/7\+M w/\+Compass &\href{http://www.hobbyking.com/hobbyking/store/__55558__Ublox_Neo_7M_GPS_with_Compass_and_Pedestal_Mount.html}{\tt Hobby\+King} \\\cline{1-2}
\end{TabularC}
\paragraph*{N\+E\+O-\/6}

\begin{TabularC}{2}
\hline
\rowcolor{lightgray}{\bf Module }&{\bf Comments  }\\\cline{1-2}
Ublox N\+E\+O-\/6\+M G\+P\+S with Compass &\href{http://www.ebay.com/itm/111585855757}{\tt e\+Bay} \\\cline{1-2}
\end{TabularC}


\subsubsection*{Serial N\+M\+E\+A}

\paragraph*{Media\+Tek}

\begin{TabularC}{2}
\hline
\rowcolor{lightgray}{\bf Module }&{\bf Comments  }\\\cline{1-2}
M\+T\+K 3329 &Tested on hardware serial at 115200 baud (default) and on softserial at 19200 baud. The baudrate and refresh rate can be adjusted using the Mini\+G\+P\+S software (recommended if you lower the baudrate). The software will estimate the percentage of U\+A\+R\+T bandwidth used for your chosen baudrate and update rate. \\\cline{1-2}
\end{TabularC}
